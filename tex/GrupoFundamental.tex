%Para este capítulo se usará la abreviatura "grf".
\chapter{Grupo fundamental}
\label{grf}

La topología algebraica comprende métodos que son significativamente distintos a los empleados hasta ahora en topología general. Intenta asignar a un espacio topológico algún invariante algebraico (por ejemplo, un grupo) y utilizar las propiedades de este invariante para obtener información sobre la topología. Este capítulo se centrará, pues, en el estudio del grupo fundamental, que es uno de estos invariantes.

\section{Homotopía}

En particular, introducimos la homotopía con el propósito de definir más adelante la noción de grupo fundamental. Dos caminos son, intuitivamente, homótopos si podemos deformar uno en el otro de forma continua.

% FALTA LA MAYORÍA DE LA SECCIÓN, EXCEPTO ESTE ÚLTIMO EJEMPLO	

\begin{exa}
	La interpolación lineal produce homotopías relativas. En efecto, consideramos la homotopía:
	\[H_s = (1-s)f + sg\]
	Si $f(a)=g(a)$ para algún $a\in\Y$, entonces resulta que:
	\[H_s(a)=(1-s)f(a)+sg(a)=f(a)=g(a)\]
	
	En particular, en $\R^n$, cualquier par de caminos cuyos extremos coincidan son homótopos con extremos fijos. De esta forma, en cualquier espacio $\X$ homeomorfo a $\R^n$ por un homeomorfismo $h:\R^n\to \X$, cualesquiera dos caminos cuyos extremos coincidan son homótopos con extremos fijos. En efecto, podemos fabricar la homotopía fácilmente: sean $\sigma,\tau$ los caminos en $\X$. Consideramos $\alpha,\beta$ caminos en $\R^n$, de forma que verifiquen que $\sigma = h\circ\alpha$, $\tau = h\circ\beta$. Como $\alpha$ y $\beta$ comparten extremos y están en $\R^n$, son homótopos con extremos fijos, y dada la homotopía $H_s$, resulta que $h\circ H_s$ es homotopía entre $\sigma$ y $\tau$.
\end{exa}

\section{Esferas}

El estudio de las homotopías en las esferas es interesante como ejemplo, y permite, basándose tan solo en lo visto en la anterior sección, demostrar un resultado no trivial.

Empezamos definiendo formalmente la esfera, para aclarar la notación.

\begin{defi}[Esfera]
	Llamamos \tbi{esfera} de dimensión $n$ y denotamos $\Sfe^n$ al subconjunto de $\R^{n+1}$:
	\[\Sfe^n = \{x\in\R^{n+1}\midc \norm{x}=1\}\subset\R^{n+1}\]
	donde $\norm{\cdot}$ es la norma euclídea. Cuando se considera como espacio topológico es con la restricción de la topología usual, si no se especifica otra.
\end{defi}

\begin{obs}
	Si bien la esfera de la definición anterior es la esfera unidad, nótese que todas las esferas de cualquier radio y centradas en cualquier punto de $\R^{n+1}$ son homeomorfas a la esfera que hemos definido.
\end{obs}

El resultado no trivial que mencionábamos en la introducción de esta sección es el siguiente:

\begin{prop}
	\label{grf_homotop_caminos_esfera}
	Dos caminos en una esfera $\Sfe^n$, $n\geq 2$, que tengan los mismos extremos son homótopos con extremos fijos.

	\begin{proof}
		% TODO: demostración
		Falta
	\end{proof}
\end{prop}

Un resultado muy relacionado ha sido un problema abierto hasta hace muy poco tiempo:

\begin{conjet}[Poincaré]
	La propiedad de la proposición \ref{grf_homotop_caminos_esfera} caracteriza a la esfera $\Sfe^3$. Esto es, si un espacio topológico verifica que cualquier par de caminos en él que tengan los mismos extremos son homótopos con extremos fijos, entonces es homeomorfo a la esfera unidad.
\end{conjet}

La versión generalizada de esta conjetura, para la esfera $\Sfe^n$, también es cierta y tiene interés. Históricamente:
\begin{itemize}
	\item Para $n=2$, el resultado se conoce desde el siglo XIX.
	\item Para $n=5$, Zeeman lo demostró en 1961.
	\item Para $n\geq 6$, Smale lo demostró en 1961.
	\item Para $n=4$, Donaldson lo demostró en 1985.
	\item Para $n=3$, la versión original de la conjetura, Perelman lo demostró en 2006. La conjetura era uno de los 7 problemas del milenio, dotados con 1.000.000\$. Perelman rechazó tanto este premio como la medalla Fields tras demostrar la conjetura.
\end{itemize}