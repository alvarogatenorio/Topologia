%Para este capítulo se usará la abreviatura "grf".
\chapter{Grupo fundamental}
\label{grf}

La topología algebraica comprende métodos que son significativamente distintos a los empleados hasta ahora en topología general. Intenta asignar a un espacio topológico algún invariante algebraico (por ejemplo, un grupo) y utilizar las propiedades de este invariante para obtener información sobre la topología. Este capítulo se centrará, pues, en el estudio del grupo fundamental, que es uno de estos invariantes.

\section{Homotopía}

En particular, introducimos la homotopía con el propósito de definir más adelante la noción de grupo fundamental. Dos caminos son, intuitivamente, homótopos si podemos deformar uno en el otro de forma continua.

% FALTA LA MAYORÍA DE LA SECCIÓN, EXCEPTO ESTE ÚLTIMO EJEMPLO	

\begin{exa}
	La interpolación lineal produce homotopías relativas. En efecto, consideramos la homotopía:
	\[H_s = (1-s)f + sg\]
	Si $f(a)=g(a)$ para algún $a\in\Y$, entonces resulta que:
	\[H_s(a)=(1-s)f(a)+sg(a)=f(a)=g(a)\]
	
	En particular, en $\R^n$, cualquier par de caminos cuyos extremos coincidan son homótopos con extremos fijos. De esta forma, en cualquier espacio $\X$ homeomorfo a $\R^n$ por un homeomorfismo $h:\R^n\to \X$, cualesquiera dos caminos cuyos extremos coincidan son homótopos con extremos fijos. En efecto, podemos fabricar la homotopía fácilmente: sean $\sigma,\tau$ los caminos en $\X$. Consideramos $\alpha,\beta$ caminos en $\R^n$, de forma que verifiquen que $\sigma = h\circ\alpha$, $\tau = h\circ\beta$. Como $\alpha$ y $\beta$ comparten extremos y están en $\R^n$, son homótopos con extremos fijos, y dada la homotopía $H_s$, resulta que $h\circ H_s$ es homotopía entre $\sigma$ y $\tau$.
\end{exa}

\section{Esferas}

El estudio de las homotopías en las esferas es interesante como ejemplo, y permite, basándose tan solo en lo visto en la anterior sección, demostrar un resultado no trivial.

Empezamos definiendo formalmente la esfera, para aclarar la notación.

\begin{defi}[Esfera]
	Llamamos \tbi{esfera} de dimensión $n$ y denotamos $\Sfe^n$ al subconjunto de $\R^{n+1}$:
	\[\Sfe^n = \{x\in\R^{n+1}\midc \norm{x}=1\}\subset\R^{n+1}\]
	donde $\norm{\cdot}$ es la norma euclídea. Cuando se considera como espacio topológico es con la restricción de la topología usual, si no se especifica otra.
\end{defi}

\begin{obs}
	Si bien la esfera de la definición anterior es la esfera unidad, nótese que todas las esferas de cualquier radio y centradas en cualquier punto de $\R^{n+1}$ son homeomorfas a la esfera que hemos definido.
\end{obs}

El resultado no trivial que mencionábamos en la introducción de esta sección es el siguiente:

\begin{prop}
	\label{grf_homotop_caminos_esfera}
	Dos caminos en una esfera $\Sfe^n$, $n\geq 2$, que tengan los mismos extremos son homótopos con extremos fijos.

	\begin{proof}
		Consideramos los caminos $\sigma,\tau:[0,1]\to \Sfe^n$, tales que $\sigma(0)=\tau(0)$ y $\sigma(1)=\tau(1)$. Para probar que son homótopos con extremos fijos, basta con probar que existe un punto $a\in\Sfe^n$ tal que $a\not\in \sigma,\tau$. Ya hemos comprobado antes que esto es suficiente (FALTA CITA). % TODO: Cita
		
		Entonces, vamos a separar la esfera en dos abiertos $U$ y $V$, de forma que $\Sfe^n=U\cup V$. Para un cierto $a\in\Sfe^n$, tal que ni $a$ ni $-a$ es uno de los extremos del camino, elegimos $U$ y $V$ tales que:
		\[\left\{\begin{array}{l}
			U = \Sfe^n\setminus\{a\} \\
			V = \Sfe^n\setminus\{-a\}
		\end{array}\right.\]
		es decir, $U$ es la esfera quitando un punto y $V$ es la esfera quitando el punto antipodal al anterior. De esta forma, ya hemos visto el hecho de que tanto $U$ como $V$ son homeomorfos a $\R^n$, por ser la esfera sin un punto (FALTA CITA, sé que lo hemos visto en clase pero no lo he encontrado). De esta forma, $U\cap V\homeo\mathbb{R}^n\setminus\{0\}$, puesto que una esfera sin un punto es homeomorfa a $\R^n$, y por tanto una esfera sin dos puntos es homeomorfa a $\R^n$ sin un punto. Además, como $n\geq 2$, sabemos que $\R^n\setminus\{0\}$ es conexo por caminos, y entonces también lo es $U\cap V$. % TODO: cita
		
		Podemos encontrar una partición $0=t_0<t_1<\dots<t_r=1$ del intervalo $[0,1]$ que verifique que $\sigma([t_{i-1},t_i])\subset U\text{ o }V$ para cada $i$. En efecto, sabemos que $\{\sigma^{-1}(U), \sigma^{-1}(V)$ es recubrimiento abierto de $[0,1]$. Por el lema de Lebesgue (lema \ref{comp_lema_lebesgue}), que podemos aplicar por ser $[0,1]$ un espacio métrico compacto (y por tanto, por la observación \ref{comp_obs_relacion_defs_compacidad}, secuencialmente compacto), para cada $x\in [0,1]$ $\exists\epsilon>0$ que cumple que $\bola(x,\epsilon)\subset\sigma^{-1}(U)\text{ o }\sigma^{-1}(V)$. Con lo cual, tomamos la partición anterior de forma que $t_i-t_{i-1}<\epsilon\;\forall i$, y entonces tenemos que, para $x$ en el intervalo:
		\[x\in [t_{i-1}, t_i]\subset B(x,\epsilon)\subset\sigma^{-1}(U)\text{ o }\sigma^{-1}(V)\]
		Por tanto, cada intervalo de la partición está en $\sigma^{-1}(U)$ o $\sigma^{-1}(V)$. Además, eliminando las divisiones innecesarias tenemos una partición que alterna estar en $\sigma^{-1}(U)$ y en $\sigma^{-1}(V)$.
		
		Sea $x_0\in U\cap V$. Vamos a homotopar los segmentos en $V$ a segmentos en $U$. Para cada $i$ tal que $\sigma\restriction_{[t_{i-1},t_i]}\subset V$, definimos $x_{i-1} = \sigma(t_{i-1})$, $x_i=\sigma(t_i)$. Resulta que $x_{i-1},x_i\subset U\cap V\homeo \R^n\setminus\{0\}$. Como ya hemos visto que $\R^n\setminus \{0\}$ es conexo por caminos, entonces necesariamente existe $\sigma_i:[t_{i-1},t_i]\to U\cap V$ camino entre $x_{i-1}$ y $x_i$. De esta forma, los caminos $\sigma\restriction_{[t_{i-1},t_i]}$ y $\sigma_i$ son homótopos en $V\homeo\R^n$, con lo cual son homótopos con extremos fijos en $V\subset\Sfe^n$. Repitiendo el argumento para cada segmento de $V$, reemplazándolo por el correspondiente $\sigma_i$, obtenemos un nuevo camino $\widetilde{\sigma}\subset U$ y claramente $\sigma\homot\widetilde{\sigma}$ con extremos fijos.
		
		Por último, repetimos todo el argumento anterior con $\tau$, y por tanto $\exists \widetilde{\tau}$ tal que $\tau\homot\widetilde{\tau}$ con extremos fijos y $\widetilde{\tau}\subset U\homeo\R^n$. Entonces, $\widetilde{\sigma}\homeo\widetilde{\tau}$ con extremos fijos en $U\subset\Sfe^n$ y, por ser $\homot$ con extremos fijos de equivalencia, $\sigma\homot\tau$ con extremos fijos.
	\end{proof}
\end{prop}

Un resultado muy relacionado ha sido un problema abierto hasta hace muy poco tiempo:

\begin{conjet}[Poincaré]
	La propiedad de la proposición \ref{grf_homotop_caminos_esfera} caracteriza a la esfera $\Sfe^3$. Esto es, si una variedad de dimensión 3 de $\R^4$ verifica que cualquier par de caminos en ella que tengan los mismos extremos son homótopos con extremos fijos, entonces es homeomorfa a la esfera unidad.
\end{conjet}

La versión generalizada de esta conjetura, para la esfera $\Sfe^n$, también es cierta y tiene interés. Históricamente:
\begin{itemize}
	\item Para $n=2$, el resultado se conoce desde el siglo XIX.
	\item Para $n=5$, Zeeman lo demostró en 1961.
	\item Para $n\geq 6$, Smale lo demostró en 1961.
	\item Para $n=4$, Donaldson lo demostró en 1985.
	\item Para $n=3$, la versión original de la conjetura, Perelman lo demostró en 2006. La conjetura era uno de los 7 problemas del milenio, dotados con 1.000.000\$. Perelman rechazó tanto este premio como la medalla Fields que se le intentó conceder por demostrar la conjetura.
\end{itemize}

% TODO: Faltan las secciones 3 y 4 (producto de caminos y grupo fundamental)
% TODO: En la sección del grupo fundamental, añadir la noción de simplemente conexo (dijo en clase que se le olvidó definirla, y esta sección es donde mejor cuadra)

\section{\ti{General nonsense}. Funtorialidad}

El objetivo principal de esta sección es aplicar las ideas de teoría de categorías a los conceptos que hemos ido viendo a lo largo de este capítulo. Así, se recomienda leer antes el anexo \ref{funt}, que contiene una brevísima introducción a los conceptos más básicos de teoría de categorías. Esta no solo define los conceptos básicos como categoría o funtor sino que puede ayudar al lector a comprender el propósito detrás de este punto de vista, que a primera vista puede parecer terriblemente abstracto.

\begin{const}[Interpretación funtorial del grupo fundamental]
	En esta sección, vamos a considerar dos categorías. La primera de ellas es la \index[general]{categoría}categoría de espacios topológicos punteados \Topp, cuyos objetos son: 
	\[\{((\X,\T),x_0)\midc (\X,\T)\text{ espacio topológico}, x_0\in\X\}\]
	y en la cual los morfismos son las funciones continuas que preservan el punto base $x_0$. La segunda de ellas la categoría de los grupos \Grp, que tiene por objetos:
	\[\{(G,\cdot)\midc (G,\cdot) \text{ grupo}\}\]
	y por morfismos los homomorfismos de grupos.
	
	Entonces, consideramos tomar el grupo fundamental como un \index[general]{funtor}funtor. Es decir, consideramos el funtor $\pi$:
	\[\xymatrix @R=0.5pc @C=0.5pc {
		& & \Topp \ar[rrr]^\pi & & & \Grp & \\
		& & (\X,x_0) \ar@{|->}[rrr] \ar[dd]_{f \text{ cont.}} \ar@<1.5ex>@{|->}[dd] & & & \pi(\X,x_0) \ar[dd]_{f^\ast}^{\text{homom.}} \ar@{}[r]|-\ni & [\sigma] \ar@{|->}[dd] \\
		[0,1] \ar[urr]^\sigma \ar[drr]_{f\circ\sigma} & & & & & & \\
		& & (\Y,y_0) \ar@{|->}[rrr] & & & \pi(\Y,y_0) \ar@{}[r]|-\ni & [f\circ\sigma] \\
	}\]
	Nótese que denotamos $f^\ast$ al homomorfismo de grupos que es imagen de una aplicación continua $f$ por el funtor $\pi$.
	Como se aprecia en el diagrama, este homomorfismo $f^\ast$ se define, para una clase de equivalencia de caminos, de la siguiente forma:
	\[\begin{split}
	f^\ast:(\X, x_0) &\to (\Y, y_0) \\
	[\sigma] &\mapsto [f\circ\sigma]
	\end{split}\]
	Hay que comprobar, entonces, que está bien definido. En efecto, consideramos la aplicación $f^\ast$ que acabamos de definir. Desde luego, si $\sigma$ es un lazo con base $x_0$, $f\circ\sigma$ es un lazo con base $y_0$. Para ver si respeta la homotopía, consideramos $\tau:[0,1]\to (\X,x_0)$ homótopo con extremos fijos con $\sigma$ a través de la homotopía $H$. Entonces, $f\circ H$ es una homotopía entre $f\cap\sigma$ y $f\cap\tau$. Por último, comprobamos que es homomorfismo, lo cual es directo comprobando que $f\circ(\sigma\ast\tau)=(f\circ\sigma)\ast (f\circ\tau)$.
	
	Además, hay que ver que es un funtor, esto es, que cumple:
	\begin{itemize}
		\item $(g\circ f)^\ast = g^\ast\circ f^\ast$. En efecto, consideramos el lazo $\sigma$ con base en $x_0$. Sabemos que:
		\[(g\circ f)\circ\sigma=g\circ(f\circ\sigma)\]
		y tomando la imagen de $\sigma$ por $(g\circ f)^\ast$:
			\[(g\circ f)^\ast([\sigma])=[(g\circ f)\circ\sigma]=[g\circ(f\circ\sigma)]=g^\ast([f\circ\sigma])=g^\ast(f^\ast([\sigma]))\]
		como queríamos comprobar.
		
		\item $(\Id_\X)^\ast = \Id_{\pi(\X, x_0)}$. Esto se verifica directamente por como hemos definido $f^\ast$.
	\end{itemize}
\end{const}

Visto todo lo necesario, podemos pasar a comprobar una serie de propiedades utilizando el punto de vista introducido en la construcción anterior. La teoría de categorías es capaz a menudo de simplificar demostraciones al considerar los objetos abstractos, lo cual es una de sus grandes fortalezas. Vamos a ver ahora una serie de proposiciones que al mismo tiempo son ejemplos, pues muestran como enfocaríamos una demostración desde este nuevo punto de vista. Sin embargo, no dejan de ser resultados interesantes por sí mismos.

\begin{prop}
	El funtor grupo fundamental manda homeomorfismos en isomorfismos de grupos. En particular, si dos espacios tienen distinto grupo fundamental, entonces no son homeomorfos.
	
	\begin{proof}
		Sean $\X,\Y$ dos espacios topológicos punteados con $x_0$, $y_0$, y $f$ un homeomorfismo entre ellos de forma que:
		\[\xymatrix @R=0.4pc {
			\X \ar[r]^f_{\text{homeo.}} & \Y \ar[r]^{g=f^{-1}} & \X \\
			x_0 \ar@{|->}[r] & y_0 \ar@{|->}[r] & x_0
		}\]
		
		Ahora, tenemos que:
		\[g^\ast\circ f^\ast = (g\circ f)^\ast = (\Id_\X)^\ast = \Id_{\pi(X)}\]
		\[f^\ast\circ g^\ast = (f\circ g)^\ast = (\Id_\Y)^\ast = \Id_{\pi(Y)}\]
		y con esto hemos comprobado que $g^\ast$ es la inversa de $f^\ast$, luego $f^\ast$ es isomorfismo.
	\end{proof}
\end{prop}

\begin{prop}
	Un \index[general]{retracto}retracto, como veremos con detalle más adelante, es un subespacio tal que existe una aplicación continua que deforma el espacio en él, como se aprecia en el siguiente esquema:
	\[\xymatrix @C=0.65pc {
		A \ar@{}[r]|\subset \ar[dr]_{\rho\restriction_A = \Id} & \X \ar[d]^{\rho\text{ retracto}} \\
		& A
	}\]
	Entonces, la imagen $\rho^\ast$ de $\rho$ por $\pi$ es un homomorfismo sobreyectivo. En particular, si el grupo fundamental del retracto $A$ es no trivial, entonces el grupo fundamental del espacio $\X$ tampoco es trivial.
	
	\begin{proof}
		Para la demostración, consideramos la imagen por $\pi$ del diagrama anterior, donde $A$ y $\X$ tienen el mismo punto base $x_0\in A$:
		\[\xymatrix {
			\pi(A,x_0) \ar[r]^{j^\ast} \ar[dr]_{(\rho\restriction_A)^\ast = \Id_{\pi(A)}} & \pi(\X,x_0) \ar[d]^{\rho^\ast} \\
			& \pi(A,x_0)
		}\]
		donde $j$ es la aplicación inclusión de $A$ en $\X$. Como por las propiedades de los funtores $\rho^\ast\circ j^\ast=(\rho\circ j)^\ast$, el diagrama anterior conmuta, es decir:
		\[\rho^\ast\circ j^\ast=\Id_{\pi(A)}\]
		
		Ahora, como $\Id_{\pi(A)}$ es biyectivo, y en particular sobreyectivo, $\rho^\ast\circ j^\ast$ también debe ser sobreyectivo, y entonces necesariamente $\rho^\ast$ lo es.
	\end{proof}
\end{prop}

Un caso donde se puede aplicar la proposición anterior es el siguiente.

\begin{exa}
	Consideramos el disco cerrado $\adher{D}_2\subset\R^2$. Queremos comprobar que el disco no se puede retraer a $A=\Sfe^1=\partial\adher{D}_2$. Como veremos más adelante, $\pi(A)=\pi(\Sfe^1)=\Z$, y ya sabemos que $\pi(\adher{D}_2)=\{1\}$, pues es un estrellado en $\R^n$. Entonces, si fuera retracto, habría un homomorfismo sobreyectivo del grupo trivial a $\Z$, y esto es imposible.
\end{exa}

\begin{prop}
	Dados dos espacios $\X,\Y$, el grupo fundamental del producto es isomorfo al producto de los grupos fundamentales. Esto es:
	\[\pi(\X\times\Y)\approx\pi(\X)\times\pi(\Y)\]
	
	\begin{proof}
		Consideramos los puntos base $x_0,y_0$ respectivamente. El esquema de la topología producto sería el siguiente:
		\[\xymatrix @R = 0.4pc{
			& \X & \\
			\X\times\Y \ar[ur]^p \ar[dr]_q & & [0,1] \ar[ul]^\sigma \ar[dl]^\tau \\
			& \Y &
		}\]
		donde $p$ y $q$ son las proyecciones. Entonces, como ya hemos hecho antes, vamos a ver la imagen de este diagrama:
		\[\xymatrix @R = 0.4pc @C = 0.5pc {
			& \pi(\X) & & & *+[r]{[p\circ (\sigma,\tau)]=[\sigma]} \\
			\pi(\X,\Y)  \ar[ru]^{p^\ast} \ar[rd]_{q^\ast} \ar[rr]^-{(p^\ast,q^\ast)} & & \pi(\X)\times\pi(\Y) \ar@{}[r]|-\colon& [(\sigma,\tau)] \ar@{|->}[ru] \ar@{|->}[rd] \ar@{|->}[r] & ([\sigma],[\tau]) \\
			& \pi(\Y) & & & *+[r]{[q\circ (\sigma,\tau)]=[\tau]}
		}\]
		
		Nótese que hemos añadido la aplicación que llega a $\pi(\X)\times\pi(\Y)$. Si comprobamos que es un isomorfismo, entonces tenemos lo que estábamos buscando. En efecto:
		\begin{itemize}
			\item Es trivialmente sobreyectivo.
			\item Es inyectivo. Para verlo, usaremos un resultado conocido que afirma que un homomorfismo es inyectivo si y solo si su núcleo es trivial. Entonces, denotando $e_X$ al elemento identidad de cada grupo $\pi(X)$, sean $\sigma,\tau$ de forma que $([\sigma],[\tau])=(e_\X,e_\Y)$. Queremos ver que necesariamente $[(\sigma,\tau)]=e_{\X\times \Y}$.
			
			Consideramos las homotopías $K_s:\sigma\homot x_0=e_\X$, $L_s:\tau\homot y_0=e_\Y$. Entonces, $(K_s,L_s):(\sigma,\tau)\homot(x_0,y_0)=e_{\X\times\Y}$ es homotopía y, por tanto, el núcleo del homomorfismo es $\{e_{\X\times\Y}\}$.
		\end{itemize}
		Por tanto, es isomorfismo.
	\end{proof}
\end{prop}

Un ejemplo de la utilidad de este resultado es el siguiente.

\begin{exa}
	Consideramos el toro $\toro$, que sabemos que es homeomorfo a $\Sfe^1\times\Sfe^1$. Entonces, cuando veamos más adelante que $\pi(\Sfe^1)=\Z$, podremos afirmar directamente que $\pi(\toro)=\Z\times\Z$.
\end{exa}

\section{Espacios recubridores. El problema de elevación}

El estudio de los espacios recubridores de otro espacio es a menudo útil para calcular grupos fundamentales, y en general tiene una gran relación con el estudio de ellos. En esta sección planteamos el problema de elevación en su versión más general, para luego particularizarlo para poder demostrar propiedades útiles.

\subsection{Espacios recubridores}

Para empezar, vamos a formalizar el concepto de espacio recubridor.

\begin{defi}[Espacio recubridor]
	Sea $p:\widetilde{\X}\to \X$ una aplicación continua y sobreyectiva. Decimos que $\widetilde{\X}$ es un \tbi[espacio!recubridor]{espacio recubridor} de $X$ si para cada $x\in\X$ existe un entorno $U_x$ que verifica que $p^{-1}(U_x)$ se puede escribir como unión disjunta de conjuntos $U_\lambda$ donde cada uno de ellos es homeomorfo a $U_x$, de forma que, para cada $\lambda$, $p\restriction_{U_\lambda}:U_\lambda\to U_x$ sea un homeomorfismo.
\end{defi}

\begin{exa}[Recubrimientos]
	\label{grf_exa_recubrimientos}
	Veamos algunos ejemplos de espacios recubridores.
	
	\begin{enumerate}
		\item $\Sfe^n$ recubre al plano proyectivo real $\proy^n$, con la siguiente aplicación:
		\[\begin{split}
		p:\Sfe^n&\to\proy^n \\
		(x_0,\dots,x_n)&\mapsto (x_0:\dotsc:x_n)
		\end{split}\]
		En efecto, tomamos un punto $(x_0:\dotsc:x_n)\in\proy^n$. Tomando un entorno abierto del punto, homeomorfo a un disco, su imagen inversa consiste en dos abiertos homeomorfos a discos alrededor de dos puntos antipodales de $\Sfe^n$. Estos son disjuntos y se verifican las condiciones.
		
		\item $\R$ recubre al círculo $\Sfe^1$, con la aplicación:
		\[\begin{split}
		p:\R&\to\Sfe^1 \\
		t &\mapsto e^{2\pi t}=(\cos 2\pi t,\sin 2\pi t)
		\end{split}\]
		En efecto, para un punto $x$ del círculo, tomamos un entorno abierto: un arco que lo contenga sin extremos. Esto es homeomorfo a un intervalo abierto, y su imagen inversa consiste en una cantidad numerable de intervalos a lo largo de la recta real, cada uno conteniendo un punto de $p^{-1}(x) = \Z + a$ para algún $a\in\R$. Como, tomando el entorno inicial lo suficientemente pequeño, todos estos intervalos son disjuntos, ya está. \qedhere
	\end{enumerate}
\end{exa}

\subsection{El problema de elevación}

Con esto, estamos ya en condiciones de formular el problema de elevación.

\begin{const}[Formulación del problema de elevación]
	\index[general]{problema de elevación}
	Consideramos el diagrama:
	\[\xymatrix{
		& \widetilde{\X} \ar[d]^p \\
		\mc{Z} \ar@{-->}[ru]^{\widetilde{H}} \ar[r]^H & \X
	}\]
	
	Aquí, $\widetilde{\X}$ es un espacio recubridor mediante la aplicación $p$. El problema de elevación pregunta por la existencia de la aplicación del diagrama $\widetilde{H}:\mc{Z}\to\X$, continua y que verifique que $p\circ\widetilde{H}=H$, es decir, que haga que el diagrama sea conmutativo. Una aplicación que verifique esto se llama \tbi{elevación}.
\end{const}

% TODO: Falta una buena parte de la sección de elevación

\section{El grupo fundamental del espacio proyectivo}

El objetivo de esta sección es calcular el grupo fundamental de $\proy^n$, el espacio proyectivo real de dimensión $n$, para $n\geq 2$. Como pronto veremos, esta demostración explota el concepto de elevación. 

\begin{theo}[Grupo fundamental de $\proy^n$]
	Para $n\geq 2$, $\pi(\proy^2)=\Z_2$.
	
	\begin{proof}
		Vamos a empezar demostrando que la esfera $\Sfe^n$ es un espacio recubridor del espacio proyectivo $\proy^n$. Vamos a repetir con un poco más de detalle el argumento que ya utilizamos en el ejemplo \ref{grf_exa_recubrimientos}. Consideramos pues la aplicación:
		\[\begin{split}
		p:\Sfe^n&\to\proy^n \\
		(x_0,\dots,x_n)&\mapsto (x_0,\dotsc,x_n)
		\end{split}\]
		Entonces, sea un punto $y\in\proy^n$. Tenemos que comprobar que para cada $x\in p^{-1}(y)$ existe un entorno $U_x$ tal que $p^{-1}(U_x)$ es union disjunta de conjuntos $U_\lambda$, donde cada $U_\lambda\homeo U_x$.
		
		De esta forma, para $y\in\proy^n$, consideramos un hiperplano $\pi$ que no lo contenga. Como el plano $\pi$ es cerrado, $U=\proy^n\setminus\pi$ es abierto, y además contiene a $x$, luego es entorno. La imagen inversa del plano $\pi$ es, para la misma aplicación definida en todo $\R^{n+1}$, un hiperplano; y de esta manera la imagen inversa de $\proy^n\setminus\pi$ por $p:\Sfe^n\to\proy^n$ son dos casquetes de la esfera. Llamamos pues a estos $U_+$, $U_-$ y desde luego son abiertos. Como la imagen inversa de $x$ son dos puntos antipodales, que llamamos $p$ y $-p$, y no están en $p^{-1}(\pi)$, entonces está cada uno en un casquete. De esta forma, como los casquetes son entornos disjuntos, basta con comprobar que $U_+,U_-\homeo U_x$. Pero resulta que $U_x = \proy^n\setminus\pi\homeo\R^n$, y $U_+,U_-\homeo\R^n$ porque son homeomorfos a discos abiertos.
		
		Visto esto, podemos considerar la elevación de caminos que vamos a utilizar. Sea $a=(1:0:\dotsc:0)$ el punto base del grupo fundamental. Entonces, tenemos el diagrama:
		\[\xymatrix @C = 0.5pc {
			&& \Sfe^n \ar[d]^p & (x_0,\dots,x_n) \ar@{|->}[d] \ar@{}[l]|-\ni \\
			[0,1] \ar[rr]^\sigma \ar[rru]^{\widetilde{\sigma}} && \proy^n & (x_0,\dotsc,x_n) \ar@{}[l]|-\ni
		}\]
		de forma que $\sigma$ es un lazo en $\proy^n$ con base en $a$, y por tanto verifica $\sigma(0)=\sigma(1)=a$. Si llamamos $p^{-1}(a)=\{\widetilde{a},\widetilde{-a}\}$, por el lema de elevación de caminos [FALTA CITA], podemos elegir $\widetilde{\sigma}$ de forma que se verifique que $\widetilde{\sigma}(0)=\widetilde{a}$. Pero como tan solo sabemos que $p\circ\widetilde{\sigma}=\sigma$, entonces $\widetilde{\sigma}(1)$ no tiene por qué ser $\widetilde{a}$, puede ser también $-\widetilde{a}$. % TODO: cita
		
		Planteamos entonces dos casos:
		\begin{enumerate}[label=\roman*]
			\item $\widetilde{\sigma}$ es un lazo en $\Sfe^n$ de base $\widetilde{a}$. Como ya vimos en la proposición \ref{grf_homotop_caminos_esfera} que $\pi(\Sfe^n)=\{1\}$, podemos escribir:
			\[\widetilde{H}_s:\widetilde{\sigma}\homot\widetilde{e}=\widetilde{a}\]
			y tomando imágenes por $p$ tenemos:
			\[H_s=p\circ\widetilde{H}_s:p\circ\widetilde{\sigma}\homot p\circ\widetilde{e}\]
			y por ser $p\circ\widetilde{\sigma}=\sigma$, tenemos que $H_s: \sigma\homot e=a$. Por tanto, $[\sigma] = 1\in\pi(\proy^n,a)$.
			
			\item $\widetilde{\sigma}$ es un camino en $\Sfe^n$ de $\widetilde{a}$ a $-\widetilde{a}$. Consideramos el camino:
			\[\widetilde{\alpha}(t)= (\cos\pi t, \sen\pi t,0,\dots, 0) \]
			que verifica que $\widetilde{\alpha}(0)=\widetilde{a}$ y $\widetilde{\alpha}(1)=-\widetilde{a}$. De nuevo, por estar en $\Sfe^n$ existe una homotopía con extremos fijos:
			\[\widetilde{H}_s:\widetilde{\sigma}\homot[\widetilde{a},-\widetilde{a}]\widetilde{\alpha}\]
			y, de nuevo tomando imágenes, obtenemos la homotopía:
			\[p\circ H_s:\sigma\homot\alpha\]
			donde los extremos fijos son $p(\widetilde{a}),p(-\widetilde{a})$, y ambos son $a$. Por tanto es homotopía de lazos. Entonces, tenemos que $[\sigma] = [\alpha]\in\pi(\proy^n,a)$.
		\end{enumerate}
		
		De esta forma, hemos comprobado que cualquier lazo de $\proy^n$ está en alguna de las dos clases de homotopía. De esta forma, podemos asegurar que $\card{\pi(\proy^n,a)}\leq 2$. Veamos pues que las clases $1$ y $[\alpha]$ son distintas.
		
		En efecto, si fueran iguales consideramos la homotopía $H_s:\alpha\homot[a] e$. Esta verifica $H_s(0)=H_s(1)\equiv a$, $H_0=\alpha$ y $H_1\equiv a$. Al elevarla, nos queda un diagrama como el siguiente:
		\[\xymatrix{
			& \Sfe^n \ar[d]^p \\
			[0,1]\times[0,1] \ar[r]^-H \ar[ru]^{\widetilde{H}} & \proy^n
		}\]
		donde la existencia de $\widetilde{H}$ está garantizada por el lema de elevación [FALTA CITA]. Además, podemos elegir que $\widetilde{H}_0=\widetilde{\alpha}$. Como ademas la homotopía es de extremos fijos, su elevación lo es, y están definidos también $\widetilde{H}_s(0) = \widetilde{a}$ y $\widetilde{H}_s(1) = -\widetilde{a}$. Pero necesariamente, por continuidad y por ser $H_1\equiv a$, $\widetilde{H}_1$ tiene que ser una constante, y tenemos $\widetilde{H}_1(0) = \widetilde{a}$ y $\widetilde{H}_1(1)=-\widetilde{a}$, lo cual es una contradicción. % TODO: cita
		
		Entonces, el grupo fundamental de $\proy^n$ tiene dos elementos, y por tanto tiene que ser $\Z_2$.
	\end{proof}
\end{theo}

\section{Grupo fundamental del círculo}

El objetivo de esta sección es calcular el grupo fundamental de $\Sfe^1$, el círculo de $\R^2$. Además, este es homeomorfo a $\proy^1$, la recta proyectiva real, con lo que ya tendremos el grupo fundamental del espacio proyectivo de cualquier dimensión.