%Para este capítulo se usará la abreviatura "cont".
\chapter{Continuidad}
\label{cont}

La continuidad es la propiedad por excelencia que queremos que nuestras funciones verifiquen. En este breve capítulo vamos a generalizar la noción de continuidad que ya conocemos y dominamos para espacios como $\R^n$, de forma que la podamos aplicar a cualquier espacio métrico conocido. La continuidad, además, será clave para definir más adelante la noción de homeomorfismo: las aplicaciones que preservan las propiedades topológicas de un espacio dado.

\section{Continuidad en un punto}

En el espacio euclídeo usual, cuando teníamos una función $f:A\subset\mb{R}^m\to\mb{R}^n$, con un punto $a\in A$, decíamos que $f$ es continua en $a$ si y solo si $\forall\varepsilon > 0\;\exists\delta > 0$ tal que si $x\in A,\norm{x-a}<\delta$, entonces $\norm{f(x)-f(a)}<\varepsilon$. Podemos reescribir esta condición como que si $x\in A\cap\bola(a,\delta)$, entonces $f(x)\in\bola(f(a),\varepsilon)$. Pero de nuevo, esto es equivalente a que para cualquier $\bola^a$ (bola centrada en $a$), $A\cap \bola^a\subset f^{-1}(\bola^{f(a)})$ para cierta $\bola^{f(a)}$.

De esta forma, vamos a proceder ahora a generalizar esta definición para espacios topológicos arbitrarios.

\begin{defi}
	Sean $\X,\mc{Y}$ espacios topológicos, $f:\X\to\mc{Y}$. $f$ es \tbi[aplicación!continua]{continua} en $x_0\in\X$ si para todo entorno $V^{f(x_0)}$ la imagen inversa $f^{-1}(V^{f(x_0)})$ es entorno de $x_0$. 
\end{defi}

\begin{obs}
	\label{cont_obs_continuidad_discreta_trivial}
	Si la topología de $\X$ es grosera, o la topología de $\mc{Y}$ es muy fina, la continuidad suele ser más fácil de comprobar. Podemos pensar en $\X$ con la topología discreta como ejemplo de lo primero y en $\mc{Y}$ con la topología trivial como ejemplo de lo segundo:
	
	\begin{enumerate}
		\item En la topología discreta, cualquier conjunto es abierto, con lo cual $\{x_0\}$ es abierto y por tanto cualquier conjunto que contenga a $x_0$ es entorno suyo. Entonces, para cualquier entorno de $f(x_0)$ su imagen inversa contendrá a $x_0$ y por lo anterior será entorno suyo. Es decir, cualquier función que nazca en $\X$ con la topología discreta es continua.
		
		\item En la topología trivial, los únicos abiertos son el vacío y el total, con lo cual dado un punto su único entorno es el total. Entonces, si $\mc{Y}$ con la topología trivial es el espacio de llegada de una función $f$, $f$ es continua, pues la imagen inversa del total es el total, y este es abierto (y por tanto entorno) en cualquier topología. \qedhere 
	\end{enumerate}
\end{obs}

De la misma forma que podemos estudiar la continuidad para unas ciertas topologías concretas, podemos estudiarla para algunas funciones concretas sin limitarnos a una topología particular. En particular, nos van a interesar la función constante y la función identidad.

\begin{obs} \
	\label{cont_obs_continuidad_cte_e_id}
	\begin{enumerate}
		\item Si $f:\X\to\mc{Y}$ es la aplicación constante $f=b$, entonces $f$ es continua con cualquier topología. En efecto, la imagen inversa de cualquier subconjunto (y en particular de cualquier entorno) de $\mc{Y}$ que contenga a $b$ es el total, que es entorno de todos los puntos.
		
		\item La continuidad de la aplicación identidad depende de los espacios topológicos sobre los que está definida, al contrario de lo que pueda parecer. En efecto, sea $f:(\X,\T_\text{discreta})\to (\X,\T_\text{trivial})$. Esta sí es continua, por la observación \ref{cont_obs_continuidad_discreta_trivial}. Sin embargo, su inversa, que también es la aplicación identidad, no es continua. Esto se sigue directamente de que, por ser la topología del espacio de llegada la discreta, $\{f(x_0)\}$ es abierto y por tanto entorno de $f(x_0)$, pero su imagen inversa es $\{x_0\}$ que con la topología trivial del espacio de salida no es entorno. \qedhere
	\end{enumerate}
\end{obs}

Ahora, veremos un par de propiedades interesantes de la continuidad en un punto.

\begin{prop}
	Dada $f:\X\to\mc{Y}$, continua en $x_0\in\X$, si $A\subset\X$ tal que $x_0\in A$, entonces $f\restriction_A:A\to\mc{Y}$ es continua en $x_0$.
	
	\begin{proof}
		Sea $V^{f(x_0)}$ un entorno de $x_0$. Como en $A$ estamos considerando la topología relativa, se verifica que $(f\restriction_A^{-1})(V^{f(x_0)}) = A\cap f^{-1}(V^{f(x_0)})$. Pero como por la continuidad de $f$ en $x_0$ tenemos que $f^{-1}(V^{f(x_0)})$ es entorno de $x_0$ en $\X$, entonces $A\cap f^{-1}(V^{f(x_0)})$ es entorno de $x_0$ en $A$.
	\end{proof}
\end{prop}

\begin{prop}
	La continuidad es una propiedad local, es decir, $f:\X\to\mc{Y}$ es continua en $x_0\in\X$ si $\exists V^{x_0}\subset\X$ entorno de $x_0$ tal que $f\restriction_{x_0}$ es continua en $x_0$.
	
	\begin{proof}
		Sea $V^{f(x_0)}$ un entorno de $f(x_0)$. Si $\exists V^{x_0}\subset\X$ entorno de $x_0$ tal que $f\restriction_{x_0}$ es continua en $x_0$, entonces $(f\restriction_{V^{x_0}})(V^{f(x_0)}) = f^{-1}(V^{f(x_0)})\cap V^{x_0}$, luego es entorno de $x_0$ en $V^{x_0}$. Entonces es entorno de $x_0$ en $\X$ y por tanto $f$ es continua.
	\end{proof}
\end{prop}

\section{Continuidad}

Tras definir la continuidad en un punto, el paso instintivo es por supuesto definir la continuidad en todo el espacio. Vamos a hacerlo y a dar una serie de definiciones equivalentes de continuidad, que abren muchas posibilidades a la hora de verificar esta propiedad.

\begin{defi}[Continuidad]
	Se dice que $f:\X\to\mc{Y}$ es \tbi[aplicación!continua]{continua} si lo es en todo punto.
\end{defi}

Intuitivamente, una aplicación continua es la que no ``rompe'' el espacio. Nos permite deformar, aplastar, girar... pero no cortar o pegar.

\begin{exa}
	Vamos a ampliar la observación \ref{cont_obs_continuidad_cte_e_id} comprobando qué tienen que verificar las topologías $\T_1$, $\T_2$ para que se cumpla que la función identidad definida de $(\X,\T_1)\to(\X,\T_2)$ es continua.
	
	El hecho de que $f$ sea continua significa que para cualquier entorno $V_2^{f(x_0)} = V^{x_0}$, la imagen inversa:
	\[f^{-1}(V_2^{x_0})=V_2^{x_0}=V_1^{x_0}\]
	cumple que es entorno de $x_0$ en el espacio $(\X,\T_1)$. Entonces, la condición necesaria y suficiente para que la función identidad entre estos dos espacios sea constante es que $\T_1\supset \T_2$.
\end{exa}

Vamos a ver ahora una serie de definiciones equivalentes de la noción de continuidad.

\begin{prop}
	Sea $f:\X\to\mc{Y}$. Entonces, las siguientes afirmaciones son equivalentes:
	
	\begin{enumerate}
		\setcounter{enumi}{-1}
		\item $f$ es continua
		\item La imagen inversa de cualquier abierto es abierta.
		\item La imagen inversa de cualquier cerrado es cerrada.
		\item $f(\adher{A})\subset\adher{f(A)}$ para cualquier subconjunto $A$ de $\X$.
		\item Existe un recubrimiento abierto arbitrario de $\X$ de la forma:
			\[\X=\bigcup\limits_{i\in I} U_i\]
			que verifica que todas las restricciones $f\restriction_{U_i}$ son continuas.
		\item Existe un recubrimiento cerrado finito de $\X$ de la forma:
			\[\X=\bigcup\limits_{i=1}^k F_i\]
			que verifica que todas las restricciones $f\restriction_{F_i}$ son continuas.
	\end{enumerate}

	\begin{proof}
		Vamos a probar las implicaciones más sencillas e ilustrativas, aunque realmente se podría hacer en cualquier orden. 
		
		\begin{enumerate}[align=left, leftmargin=*]
			\item[\fbox{$(0)\implies (1)$}] Sea un abierto $U\subset\mc{Y}$. Por ser abierto, es entorno de todos sus puntos. Pero para cada punto $f(x_0)\in U$, por ser $f$ continua, la imagen inversa de $U$ es también entorno de $x_0$. De esta forma, para cualquier $x_0\in f^{-1}(U)$, se cumple que $f^{-1}(U)$ es entorno de $x_0$, y por tanto $f^{-1}(U)$ es abierto.
			
			\item[\fbox{$(1)\implies (0)$}] Sea $V^{f(x_0)}$ un entorno en $\mc{Y}$. Por definición, contiene un abierto $U$ tal que $f(x_0)\in U$. Ahora, por hipótesis, $f^(-1)(U)$ es abierto, y como $x_0\in f^{-1}(U)\subset f^{-1}(V^{f(x_0)})$, se verifica que $f^{-1}(V^{f(x_0)})$ contiene un abierto que contiene a $x_0$ y por tanto es entorno.
			
			\item[\fbox{$(1)\Longleftrightarrow (2)$}] $F\subset\mc{Y}$ es cerrado $\iff$ $\mc{Y}\setminus F$ es abierto. Pero entonces por hipótesis $f^{-1}(\mc{Y}\setminus F)$ es abierto, y $f^{-1}(\mc{Y}\setminus F)=\X\setminus f^{-1}(F)$, luego $f^{-1}(F)$ es cerrado. La otra implicación es análoga.
			
			\item[\fbox{$(2)\implies (3)$}] Basta con ver que cualquier $A$ verifica que $f(\adher{A})\subset \adher{f(A)}$ o, lo que es equivalente, $\adher{A}\subset f^{-1}(\adher{f(A)})$. Sin embargo, la imagen inversa del cerrado $\adher{f(A)}$ es cerrada, con lo que basta con demostrar que $A\subset f^{-1}(\adher{f(A)})$, ya que $\adher{A}$ es el menor cerrado que contiene a $A$. Por tanto, simplemente:
			\[A\subset f^{-1}(f(A))\subset f^{-1}(\adher{f(A)})\]
			porque $f(A)\subset\adher{f(A)}$.
			
			\item[\fbox{$(3)\implies (2)$}] Sea $F\subset\mc{Y}$ cerrado. Queremos probar que $G=f^{-1}(F)$ también lo es, y tenemos, por hipótesis y por las propiedades de la imagen inversa:
			\[f(\adher{G})\subset\adher{f(G)}=\adher{f(f^{-1}(F))}\subset\adher{F}=F\]
			pero entonces $\adher{G}\subset f^{-1}(F)=G$, luego $\adher{G}=G$ y entonces $G$ es cerrado por la proposición \ref{etop_prop_cerradosAdher}.
						
			\item[\fbox{$(0)\implies (4)$}] Trivial, con el recubrimiento cuyo único abierto es $\mc{X}$.
			
			\item[\fbox{$(4)\implies (1)$}] Vamos a aprovechar que ya hemos demostrado $(0)\implies (1)$. Entonces, sea $U\subset\mc{Y}$ un abierto. Lo podemos escribir como unión de abiertos de forma que cada uno de ellos esté en un $U_i$, de la siguiente forma:
			\[U=\bigcup\limits_{i\in I} (U_i\cap U)\]
			
			Ahora, la imagen inversa de $U$ es:
			\[f^{-1}(U)=f^{-1}\left(\bigcup\limits_{i\in I} (U_i\cap U)\right)=\bigcup\limits_{i\in I} (f^{-1}(U_i\cap U))\]
			pero como $U_i\cap U\subset U_i$ y $f\restriction_{U_i}$ es continua, entonces cada una de estas imágenes inversas es continua, y la imagen inversa de $U$ también lo es.
			
			\item[\fbox{$(0)\implies (5)$}] Trivial, con el recubrimiento cuyo único cerrado es $\mc{X}$.
			
			\item[\fbox{$(5)\implies (2)$}] Es análogo a $(4)\implies (1)$. \qedhere
		\end{enumerate}
	\end{proof}
\end{prop}

Terminamos con una propiedad fundamental de las funciones continuas: la composición respeta la continuidad.

\begin{prop}
	Sean $f:\X\to\mc{Y}$, $g:\mc{Y}\to\mc{Z}$, con $f,g$ continuas. Entonces, $h = g\circ f$ es continua.
	
	\begin{proof}
		Esta es una consecuencia casi directa de la definición. En efecto, si consideramos $x_0\in\X$ y sus imágenes:
		\[x_0\mapsto f(x_0)\eqqcolon y_0\mapsto g(f(x_0))\eqqcolon z_0\]
		entonces basta con estudiar los entornos. En efecto, si $V^{z_0}$ es entorno de $z_0$, por la continuidad de $g$ su imagen inversa es un entorno $V^{y_0}$ en $\mc{Y}$ y, ahora, por la continuidad de $f$, la imagen inversa de este último es un entorno de $z_0$.
	\end{proof}
\end{prop}

\section{Homeomorfismo. Homeomorfismo local}
\label{cont_homeomorfismos}
En esta sección trataremos la idea de homeomorfismo en espacios topológicos. Esta idea va a adquirir una gran importancia para nosotros ya que es la que nos va a permitir caracterizar espacios distintos como "similares" desde un punto de vista topológico (que preservan sus propiedades topológicas, vemos que continúa la idea de isomorfismo que hemos visto en otras asignaturas).


Además, más adelante nos proporcionará una serie de características comunes entre espacios homeomorfos(apertura, conexión, compacidad\dots).

\label{cont_def_homeomorfismo}

\begin{defi}[Homeomorfismo]
	Un \tbi{homeomorfismo} entre espacios topológicos $f\colon X\rightarrow Y$ es una biyección continua con inversa continua.
	
	Si existe un homeomorfismo entre dos espacios X e Y se dice que estos son \tbi{homeomorfos}.
\end{defi}

Hagamos ahora unas pequeñas observaciones antes de pasar a una serie de ejemplos que nos permitan asentar estos conceptos.

\label{cont_obs_defHomeomorfismo}
\begin{obs}
	Como vemos en la definición \ref{cont_def_homeomorfismo}, no nos basta únicamente con que nuestra aplicación $f$ sea biyectiva (y de este modo tenga inversa), sino que además exigimos esta inversa sea continua. 
	
	Esta continuidad en ambos sentidos del homeomorfismo nos va a resultar muy útil como veremos más adelante, dado que las muchas propiedades (abiertos, cerrados\dots) se van a transladar entre los dos espacios homeomorfos que nos proporciona $f$.
\end{obs}

Ahora pasamos a observar una serie de funciones homeomorfas y no homeomorfas, para comprender las diferencias entre ambas y así afianzar la definición.
\label{etop_exa_homeomorfismos}
\begin{exa}[Homeomorfismos]
	
	\begin{enumerate}
		\item La función $Id\colon(X,\tau_{dis})\rightarrow(X,\tau_{triv})$ verifica ser continua y biyectiva, pero como vimos en el ejemplo \ref{cont_obs_continuidad_discreta_trivial} su inversa no es continua, por lo que no se tratará de un homeomorfismo. 
		
		\item %añadir dibujos!!!
		Vemos como cualquier función $f$ que mande $X$ a $Y$ no es homeomorfismo, a pesar de que $f$ pueda ser biyectiva y continua.
		
		Esto no podemos afirmarlo con nuestros conocimientos actuales, pero adelantándonos en el temario (buscamos al introducir este ejemplo tan solo aumentar el interés del lector por los homeomorfismos, no buscamos que se comprenda completamente) lo sabremos ya dado el punto $o$ (aquel que se manda al origen en Y) tenemos que:
		
		$\forall V^o$ pinchado (es decir, quitamos $o$) $V^o$ tiene al menos 4 componentes conexas.%añadir dibujo
		
		$\exists W^o$ pinchado  con 2 componentes conexas.%añadir dibujo
		
		Por lo tanto, al no mantenerse la cantidad de componentes conexas entre $X$ e $Y$ se verifica que $f$ no es homeomorfismo.
		
		\item %añadir dibujos(geogebra)
		Veamos como los conjuntos anteriores son homeomorfos entre si.
		
		 Tomando un recubrimiento por cerrados del primero (el que tiene por cerrados los tres segmentos) y haciendole corresponder mediante $f $con los tres segmentos de nuestro segundo conjunto, vemos como cada una de sus restricciones es continua, y f de este modo continua (hacemos este camino a la inversa para ver que la inversa también lo es y así probar que es homeomorfismo). 
		
		Este ejemplo nos puede resultar ilustrativo de que nos basta con encontrar una $f$ que homeomorfa entre nuestros espacios, pero no en todo el plano (lo cual es mucho más ambicioso).
	\end{enumerate}
\end{exa}

Una vez definido el concepto de homeomorfismo y vista a través de los ejemplos su gran fuerza, vamos a pasar al concepto de homeomorfismo local, el cual, a pesar de ser una relación más débil que la que proporciona el homeomorfismo, también será muy utilizado a lo largo de la asignatura.

\label{cont_def_homeomorfismoLocal}
\begin{defi}[Homeomorfismo Local]
	Sea $f:X\rightarrow Y$ aplicación entre espacios topológicos y $x_0\in X$. Se dice que $f$ es \tbi[homeomorfismo!local]{homeomorfismo local} en $x_0$ si $x_0$ tiene un entorno abierto $U^{x_0}$ tal que $f(U^{x_0})$ es entorno abierto de $f(x_0)$ en $Y$ y se tiene que $f|_{x_0}:U^{x_0}\rightarrow f(U^{x_0})$ es homeomorfismo.
\end{defi}

De esta definición se desprende que todo un homeomorfismo entre dos espacios es en un homeomorfismo local en todos sus puntos. Este resultado resulta evidente, pero su contrarreciproco (no homeomorfo local implica no homeomorfo) nos puede resultar enormemente útil ya que es mucho más sencillo estudiar  el homeomorfismo local al global.
Vemos ahora algunos ejemplos de homeomorfismos locales.

\label{cont_exa_homeomorfismoLocal}
\begin{exa}
	%falta dibujo esfera
	Vemos como podemos construir entre la esfera y el plano un homeomorfimo local en cada uno de sus puntos, por lo que podemos decir de estos que resultan localmente homeomorfos.
	Resulta de interés destacar que no es necesario que la aplicación sea la misma en todo el espacio, sino que podemos tomar una distinta para cada punto del espacio.
	
	Tal y como se ve en el ejemplo, 
	$f(x0,y0,z0)=(x0,y0)$ siendo homeomorfismo en su restricción a $U^{x_0}$
	$f(x1,y1.z1)=(y1,z1)$ siendo del mismo modo homeomorfismo en su restricción
\end{exa}

Una vez vistas ambas definiciones pasamos a ver una serie de propiedades y observaciones propias de los homeomorfismos (globales), pero que también nos valdrán para la restricción homeomorfa de los locales (dado que en ella por definición la aplicación es homeomorfa)
\label{etop_obs_homeomorfismo}
\begin{obs}
	\begin{enumerate}
		\item 
		Sea $f:X\rightarrow Y$ aplicación biyectiva.
		Que $f$ sea continua es equivalente a que si $U\in Y$ es abierto, $f^{-1}(U)$ también lo será. Del mismo modo, es equivalente a que si $U\in Y$ es cerrado, $f^{-1}(U)$ también lo será.
		
		Igualmente, que $f^{-1}$ sea continua es equivalente a que si $U\in X$ es cerrado, $(f^{-1})^{-1}(U)=f(U)$ también lo será. Así también, es equivalente a que si $U\in X$ es cerrado, $f(U)$ también lo será.
		
		Vemos como la si se verifican ambas (continuidad de $f$ y de su inversa) será homeomorfismo, de modo que hemos encontrado una caracterización de estos en función de las imágenes directa e inversa de los abiertos (o cerrados).
		
		\item
		Definimos las siguientes propiedades de aplicaciones:
		Una aplicación $f$ no necesariamente biyectiva se llama \tbi[aplicación!abierta]{abierta} cuando la imagen de abiertos es un abierto (es decir, cuando $f(U)$ es abierta si $U$ es abierto).
		Una aplicación $f$ no necesariamente biyectiva se llama \tbi[aplicación!cerrada]{cerrada} cuando la imagen de abiertos es un abierto (es decir, cuando $f(U)$ es abierta si $U$ es abierto).
		
		Vemos como al no ser $f$ biyectiva, no tienen porque ser equivalentes. (Dejamos para el lector el comprobar que en caso de ser biyectiva si lo es, utilizando las distintas caracterizaciones de continuidad de aplicaciones).
		
		\item
		Sean los espacios $X$ compacto, $Z\subset X$ cerrado e $Y$ tomando en $Y$ la topología $\T_2$.
	
		 Tomamos $f: X\rightarrow Y$ aplicación continua, entonces tendremos que:
		
		Como probaremos más adelante, cerrado en compacto implica compacto, por lo que $Z$ será cerrado en $X$.
		Como $f$ es continua, $f(Z)$ será compacto en $Y$ y como estamos trabajando con la topología $\T_2$ será cerrado.
		
		Así, tenemos por lo tanto que si $f$ es biyección, su inversa (al ser $f$ cerrada) será continua. Por lo tanto, afirmamos que toda aplicación continua en compactos es homeomorfismo.
		Como se habrá percatado el lector, aun no disponemos siquiera de una definición de compacidad, por lo que esta demostración queda aun fuera de su alcance, pero hemos decidido incluirla para así ver resultados interesantes sobre homeomorfismos. Recomendamos que una vez que se hayan dado los contenidos relacionados con compacidad se regrese a esta sección para así comprobar la comprensión de estos argumentos.
	\end{enumerate}
\end{obs}
