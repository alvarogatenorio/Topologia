\chapter{Continuidad}
\label{cont}

La continuidad es la propiedad por excelencia que queremos que nuestras funciones verifiquen. En este breve capítulo vamos a generalizar la noción de continuidad que ya conocemos y dominamos para espacios como $\R^n$, de forma que la podamos aplicar a cualquier espacio métrico conocido. La continuidad, además, será clave para definir más adelante la noción de homeomorfismo: las aplicaciones que preservan las propiedades topológicas de un espacio dado.

\section{Continuidad en un punto}

En el espacio euclídeo usual, cuando teníamos una función $f:A\subset\mb{R}^m\to\mb{R}^n$, con un punto $a\in A$, decíamos que $f$ es continua en $a$ si y solo si $\forall\varepsilon > 0\;\exists\delta > 0$ tal que si $x\in A,\norm{x-a}<\delta$, entonces $\norm{f(x)-f(a)}<\varepsilon$. Podemos reescribir esta condición como que si $x\in A\cap\bola(a,\delta)$, entonces $f(x)\in\bola(f(a),\varepsilon)$. Pero de nuevo, esto es equivalente a que para cualquier $\bola^a$ (bola centrada en $a$), $A\cap \bola^a\subset f^{-1}(\bola^{f(a)})$ para cierta $\bola^{f(a)}$.

De esta forma, vamos a proceder ahora a generalizar esta definición para espacios topológicos arbitrarios.

\begin{defi}
	Sean $\X,\mc{Y}$ espacios topológicos, $f:\X\to\mc{Y}$. $f$ es \tbi{continua} en $x_0\in\X$ si para todo entorno $V^{f(x_0)}$ la imagen inversa $f^{-1}(V^{f(x_0)})$ es entorno de $x_0$. 
\end{defi}

\begin{obs}
	\label{cont_obs_continuidad_discreta_trivial}
	Si la topología de $\X$ es grosera, o la topología de $\mc{Y}$ es muy fina, la continuidad suele ser más fácil de comprobar. Podemos pensar en $\X$ con la topología discreta como ejemplo de lo primero y en $\mc{Y}$ con la topología trivial como ejemplo de lo segundo:
	
	\begin{enumerate}
		\item En la topología discreta, cualquier conjunto es abierto, con lo cual $\{x_0\}$ es abierto y por tanto cualquier conjunto que contenga a $x_0$ es entorno suyo. Entonces, para cualquier entorno de $f(x_0)$ su imagen inversa contendrá a $x_0$ y por lo anterior será entorno suyo. Es decir, cualquier función que nazca en $\X$ con la topología discreta es continua.
		
		\item En la topología trivial, los únicos abiertos son el vacío y el total, con lo cual dado un punto su único entorno es el total. Entonces, si $\mc{Y}$ con la topología trivial es el espacio de llegada de una función $f$, $f$ es continua, pues la imagen inversa del total es el total, y este es abierto (y por tanto entorno) en cualquier topología. \qedhere 
	\end{enumerate}
\end{obs}

\begin{obs}
	Hay algunas funciones muy simples cuya continuidad se puede estudiar de forma más o menos general.
	
	\begin{enumerate}
		\item Si $f:\X\to\mc{Y}$ es la aplicación constante $f=b$, entonces $f$ es continua con cualquier topología. En efecto, la imagen inversa de cualquier subconjunto (y en particular de cualquier entorno) de $\mc{Y}$ que contenga a $b$ es el total, que es entorno de todos los puntos.
		
		\item La continuidad de la aplicación identidad depende de los espacios topológicos sobre los que está definida, al contrario de lo que pueda parecer. En efecto, sea $f:(\X,\T_\text{discreta})\to (\X,\T_\text{trivial})$. Esta sí es continua, por la observación \ref{cont_obs_continuidad_discreta_trivial}. Sin embargo, su inversa, que también es la aplicación identidad, no es continua. Esto se sigue directamente de que, por ser la topología del espacio de llegada la discreta, $\{f(x_0)\}$ es abierto y por tanto entorno de $f(x_0)$, pero su imagen inversa es $\{x_0\}$ que con la topología trivial del espacio de salida no es entorno.
	\end{enumerate}
\end{obs}

Ahora, veremos un par de propiedades interesantes de la continuidad en un punto.

\begin{prop}
	Dada $f:\X\to\mc{Y}$, continua en $x_0\in\X$, si $A\subset\X$ tal que $x_0\in A$, entonces $f\restriction_A:A\to\mc{Y}$ es continua en $x_0$.
	
	\begin{proof}
		Sea $V^{f(x_0)}$ un entorno de $x_0$. Como en $A$ estamos considerando la topología relativa, se verifica que $(f\restriction_A^{-1})(V^{f(x_0)}) = A\cap f^{-1}(V^{f(x_0)})$. Pero como por la continuidad de $f$ en $x_0$ tenemos que $f^{-1}(V^{f(x_0)})$ es entorno de $x_0$ en $\X$, entonces $A\cap f^{-1}(V^{f(x_0)})$ es entorno de $x_0$ en $A$.
	\end{proof}
\end{prop}

\begin{prop}
	La continuidad es una propiedad local, es decir, $f:\X\to\mc{Y}$ es continua en $x_0\in\X$ si $\exists V^{x_0}\subset\X$ entorno de $x_0$ tal que $f\restriction_{x_0}$ es continua en $x_0$.
	
	\begin{proof}
		Sea $V^{f(x_0)}$ un entorno de $f(x_0)$. Si $\exists V^{x_0}\subset\X$ entorno de $x_0$ tal que $f\restriction_{x_0}$ es continua en $x_0$, entonces $(f\restriction_{V^{x_0}})(V^{f(x_0)}) = f^{-1}(V^{f(x_0)})\cap V^{x_0}$, luego es entorno de $x_0$ en $V^{x_0}$. Entonces es entorno de $x_0$ en $\X$ y por tanto $f$ es continua.
	\end{proof}
\end{prop}