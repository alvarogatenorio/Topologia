%Para este capítulo se usará la abreviatura "comp".
\chapter{Compacidad}
\label{comp}

La compacidad en espacios topológicos es una noción bastante elaborada, y su generalización llevó a los matemáticos bastante tiempo. Desde principios del siglo XX la idea que buscaban era generalizar para espacios topológicos arbitrarios las propiedades de los intervalos cerrados y acotados $[a,b]$ de $\R$ que permitía demostrar teoremas como el del valor medio o la continuidad uniforme. Surgieron así distintos tipos de compacidad, tales como la compacidad por punto límite, la compacidad numerable,... pero no siendo estas las más adecuadas, se acabo por formalizar en términos de abiertos (en concreto, como recubrimientos de abiertos). Eso dio lugar a la definición actual.

\section{Definición y propiedades}

\begin{defi}[Recubrimiento y recubrimiento de abiertos]
	Una colección $\A$ de subconjuntos del espacio $\X$ se dice que es un \tbi{recubrimiento} si la unión de los elementos de $\A$ cubre $\X$, es decir,
	\[\X = \bigcup_{A_i \in \A}A_i\]
	Como su propio nombre indica, $\A$ será un \tbi[recubrimiento!abierto]{recubrimiento abierto} de $\X$ si es un recubrimiento de $\X$ formado por conjuntos abiertos.
\end{defi}

\begin{defi}[Compacto]
	Diremos que un espacio $\X$ es \tbi{compacto} \indexg{espacio!compacto|see{compacto}} si de cada recubrimiento abierto $\A$ de $\X$ podemos extraer un subrecubrimiento finito que también recubre $\X$.
\end{defi}

Veamos algunos ejemplos ya conocidos.

\begin{exa}[Miscelánea de ejemplos] Presentamos algunos ejemplos de conjuntos compactos y no compactos esbozando una demostración en cada uno.
	\begin{enumerate}
		\item Todo espacio topológico finito es trivialmente compacto.
		\item El intervalo $[a,b]$ es compacto. Un posible argumento para demostrar este hecho se basa en el razonamiento por contradicción. Si dividimos el intervalo en dos. Alguna de las dos mitades no será ``cubrible'' con una cantidad finita de abiertos de un recubrimiento inicial. Tomando la mitad problemática y repitiendo el argumento obtenemos una familia de intervalos cerrados y encajados sobre la que aplicar el teorema de Cantor, llegando así a una contradicción (se dejan los detalles al lector). 
		\item El intervalo $(a,b)$ no es compacto. Basta con tomar un recubrimiento del tipo $(a+\frac{1}{n},b)$ y demostrar que no se puede extraer ningún subrecubrimiento finito.
		\item La recta real $\R$ no es compacta. En efecto, si tomamos un recubrimiento por abiertos, por ejemplo $\A=\{(n,n+2) \midc n \in \Z\}$ podemos demostrar fácilmente que todo subrecubrimiento finito no cubre a $\R$. Una forma de demostrarlo es tomar el mínimo de los extremos izquierdos y el máximo de los extremos derechos y ver que aún quedan muchos puntos por cubrir.
		\item El subespacio $\X = \{0\} \cup \{\frac{1}{n} \midc n \in \N\}$ es compacto. En efecto, dado un recubrimiento abierto $\A$, existe un abierto $\U$ de $\A$ que contiene al $0$. De hecho, $\U$ contiene todos los puntos de la forma $\frac{1}{n}$ salvo un número finito de ellos (esto es consecuencia de la convergencia de la sucesión $\{\frac{1}{n}\}_{n=1}^\infty$). Para cada uno de estos puntos cogemos un abierto del recubrimiento. Por lo tanto tenemos un subrecubrimiento finito. \qedhere
	\end{enumerate}
\end{exa}
En la siguiente proposición veremos que la noción de ser compacto es independiente del espacio ambiente, es decir, si un conjunto $K$ es compacto lo es independientemente de si puede ser visto como subespacio de otro más grande o no.

\begin{prop}[Independencia del ambiente]
		Sea $\Y$ un subespacio de $\X$. Entonces $\Y$ es compacto si y sólo si cada recubrimiento de $\Y$ por abiertos de $\X$ contiene una subcolección finita que cubre $\Y$.
\end{prop}
\begin{proof} Demostremos ambas implicaciones.
	\begin{enumerate}
		\item[\bra] Supongamos que $\Y$ es compacto y que $\A=\{A_i\}_{i \in J}$ es un recubrimiento de $\Y$ por abiertos de $\X$. Entonces la colección formada por
		\begin{equation*}
		\{A_i \cap \Y \midc i \in J\}
		\end{equation*} es un recubrimiento de $\Y$ por conjuntos abiertos de $\Y$, y como $\Y$ es compacto, existe un subrecubrimiento finito de $\Y$ de la forma
		\begin{equation*}
		\{\A_{i_1} \cap \Y, ... , \A_{i_n} \cap \Y\} 
		\end{equation*}
		luego $\{\A_{i_1}, ... , \A_{i_n}\}$ es un subrecubrimiento finito de abiertos de $\X$ que cubre a $\Y$.
		\item[\bla] Sea $\A'=\{A'_{i}\}_{i\in J}$ un cubrimiento de $\Y$ por abiertos de $\Y$. Para cada $i$ podemos elegir un conjunto $A_i$ abierto en $\X$ tal que 
		\begin{equation*}
		A'_i=A_i \cap \Y
		\end{equation*}
		La colección formada por estos $A_i$ a la que llamaremos $\A$ es un recubrimiento de $\Y$ por abiertos de $\X$. Por hipótesis, existe algún subrecubrimiento finito $\{A_{i_1},...,A_{i_n}\}$ que cubre $\Y$. Entonces $\{A'_{i_1},...,A'_{i_n}\}$ es una subrecubrimiento finito de $\Y$, luego $\Y$ es compacto. \qedhere
	\end{enumerate}
\end{proof}
Veamos que podríamos haber definido la compacidad con cerrados en lugar de con abiertos, de hecho, a veces puede resultar bastante útil.
\begin{obs}[Definición alternativa]
	Podemos expresar la compacidad mediante cerrados ``dualizando'' la definición mediante complementación. Veámoslo.
	
	Por definición $\X$ es compacto si dado un recubrimiento abierto de $\X$, es decir $\X = \bigcup_{i\in I}U_i$, entonces existe un subrecubrimiento finito, o sea $\X = U_{i_1} \cup \dots \cup U_{i_n}$.
	
	Definamos $F_i:=\X\setminus U_i$ y tomando complementarios (aplicando las leyes de De Morgan) obtenemos que, equivalentemente $\X$ es compacto si y solo si dada una familia de cerrados con intersección vacía, es decir, $\emptyset = \bigcap_{i}F_i$, entonces podemos extraer una subfamilia finita que ya tenga intersección vacía, o sea, $\emptyset = F_{i_1} \cap \dots \cap F_{i_n}$.
	
	Para rizar el rizo podemos tomar el contrarrecíproco, quedando que $\X$ es compacto si y solo si dada una familia de cerrados donde todas las intersecciones finitas son no vacías entonces la intersección de toda la familia también es no vacía.
\end{obs}

De ahora en adelante revisaremos algunas consecuencias ya conocidas de la compacidad. Sin embargo, adoptaremos el nuevo alfabeto topológico para ello, viendo como algunas propiedades ciertas en espacios métricos se pierden aquí como lágrimas en la lluvia. Es hora de morir\dots

\begin{prop}[Cerrado en compacto es compacto]\label{comp_prop_cerradoCompactoCompacto}
	Sea $\X$ un espacio topológico compacto, e $\Y$ un subespacio suyo. Si $\Y$ es cerrado entonces es compacto.
\end{prop}
\begin{proof}
	Es una adaptación a la demostración del lema \ref{sep_lem_cerradoLindelofLindelof}.
	
	Tomamos un recubrimiento de $\Y$ tal que $\Y \subset \bigcup_{i\in J}U_i$. Como $\Y$ es cerrado $\X\setminus \Y$ es abierto. Esto implica que $\{U_i\}_{i\in J}\cup (\X\setminus \Y)$ es un recubrimiento por abiertos de $\X$.
	
	Como $\X$ es compacto podemos extraer un subrecubrimiento finito al que, sin pérdida de generalidad le añadimos el abierto $\X\setminus \Y$. Es decir
	\begin{equation*}
		\X = (\X\setminus\Y) \cup U_{i_1} \cup ... \cup U_{i_n}
	\end{equation*}
	De donde automáticamente se concluye que $\Y \subset U_{i_1} \cup ... \cup U_{i_n}$, luego $\Y$ es compacto.
\end{proof}

A continuación demostraremos que la compacidad y la continuidad se comportan bien.

\begin{prop}[Compacidad y continuidad]
	\label{comp_prop_compContComp}
	Sea $\X$ compacto y $f:\X \to \Y$ una aplicación continua. Entonces, $f(\X)$ es compacto.
\end{prop}
\begin{proof}
	De nuevo, esta demostración es análoga a la de la proposición \ref{sep_prop_lindelofContLindelof}.
	
	Nótese que $f:\X\to f(\X)$ es sobreyectiva. Sea $\{U_i\}_{i\in J}$ un recubrimiento abierto de $f(\X)$ tal que $f(\X)=\bigcup_{i\in J}U_i$. Tomando imágenes inversas tenemos que $\X=\bigcup_{i\in J}f^{-1}(U_i)$. 
	
	Como $f$ es continua los $f^{-1}(U_i)$ son abiertos, luego tenemos un recubrimiento abierto de $\X$. Como $\X$ es compacto, podemos extraer un subrecubrimiento finito. $\X=\bigcup_{i=1}^nf^{-1}(U_i)$.
	
	Tomando imágenes directas, se tiene que $f(\X)=\bigcup_{i=1}^nU_i$, luego $f(\X)$ es compacto.
\end{proof}

Una de las cosas que tenemos grabadas a fuego en nuestras frágiles mentes es que los compactos son cerrados y acotados. No obstante, la acotación es una idea que lleva implícita la noción de distancia, cosa que estos mundos oscuros no existe, por tanto debemos desterrarla al olvido para siempre.

Sin embargo, la idea de cerrado es perfectamente legítima en los espacios topológicos lo cual nos induce a pensar que lo que era cierto es espacios métricos lo seguirá siendo en este contexto. Para nuestra desgracia, esto no es así salvo que pidamos condiciones adicionales de regularidad.

\begin{prop}[Compacidad y clausura]\label{comp_prop_compCerrado}
	Si $\Y$ es un subespacio compacto de un espacio  \hausdorff, entonces $\Y$ es cerrado.
\end{prop}
\begin{proof}
	Denotando por $\X$ al espacio \hausdorff ambiente, demostraremos que $\X\setminus \Y$ es abierto viendo que es entorno de todos sus puntos.
	
	Sea un punto $a\in\X\setminus \Y$, veamos que hay un entorno abierto de $a$ contenido en $\X\setminus \Y$.
	
	En efecto, dado un punto $y \in \Y$, como $\X$ es \hausdorff podemos tomar entornos abiertos disjuntos $U_a^y$ y $V_y$ de los puntos $a$ e $y$ respectivamente.
	
	Es claro que la colección $\{V_y \midc y \in \Y\}$ es un recubrimiento abierto de $\Y$, y por ser este compacto existirá un subrecubrimiento finito $\{V_{y_1},\dots,V_{y_n}\}$.
	
	Es evidente que $V:=\bigcup_{i=1}^nV_{y_i}$ contiene a $\Y$. Además, $V$ y $U := \bigcap_{i=1}^nU_a^{y_i}$ son disjuntos.
	
	Por lo tanto $U$ es entorno de $a$ contenido en $\X\setminus\Y$.
\end{proof}
\begin{cor}[Compacidad y separación]
	\label{comp_obs_compSep}
	Un compacto $\Y$ y un punto $a\in \X\setminus \Y$, en un espacio topológico $\X$ que es \hausdorff, tienen entornos disjuntos.
\end{cor}
\begin{proof}
	Es el contenido de la demostración de la proposición \ref{comp_prop_compCerrado}.
	
	Nótese la importancia de que $\X$ sea \hausdorff. Obsérvese que en $\X=\{a,b\}$ con la topología del punto $\T_a$ todo es compacto por ser finito y, sin embargo, $\{a\}$ no es cerrado.
\end{proof}
Continuemos nuestra noble andadura por estos senderos que sin duda alguna nos llevarán o bien a la tumba o bien a la más profunda de las demencias.
\begin{prop}[Weierstrass]\label{comp_prop_weierstrass}
	Sea $\X$ compacto y $f:\X\to\R$ una aplicación continua entonces $f$ alcanza el máximo y el mínimo.
\end{prop}
\begin{proof}
	Como $f(\X)$ es compacto en $\R$ por ser $f$ continua, tenemos que $f(\X)$ es cerrado y acotado, luego $f(\X)$ tendrá supremo e ínfimo que pertenecen al conjunto por formar parte de su adherencia.
\end{proof}

Cavando en las tumbas de los teoremas muertos nos encontramos con el viejo y omnipresente teorema de Bolzano--Weierstrass, del que en este mundo aún queda un pequeño remanente.

\begin{prop}[Bolzano--Weierstrass]
		Sea $\X$ compacto e $\Y$ un subespacio infinito de $\X$.
		
		Entonces $\Y$ tiene algún punto de acumulación en $\X$.
\end{prop}
\begin{proof}
	Si $\Y$ no tuviese ningún punto de acumulación en $\X$, entonces todo punto de $\X$ tendría un entorno abierto $U_x$ que intersecado con $\Y$ fuera a lo sumo él mismo.
	
	Consideramos el recubrimiento abierto formado por los $U_x$. Como $\X$ es compacto, podemos extraer un subrecubrimiento finito tal que $\Y\subset \X=\bigcup_{i=1}^nU_{x_i}$. Por tanto \[\Y=\Y\cap\bigcup_{i=1}^nU_{x_i}=\bigcup_{i=1}^n\Y\cap U_{x_i}\subset\{x_{i_1},\dots,x_{i_n}\}\]
	Lo cual es absurdo por ser $\Y$ infinito.
\end{proof}

\begin{obs}[Recíproco]
	Al contrario de lo que ocurría en espacios métricos, en general, el recíproco de la proposición anterior no se cumple. Se anima al lector a encontrar un contraejemplo.
\end{obs}

Por último, y antes de introducir una serie de definiciones dedicadas a alimentar la curiosidad del lector, hagamos una pequeña observación de gran importancia en forma de lema.

\begin{lem}[Aplicaciones cerradas]
	Si tenemos una función continua $f: \X\to \Y$, donde $\X$ es compacto e $\Y$ es \hausdorff, entonces la aplicación es cerrada.
\end{lem}
\begin{proof}
	En efecto, sea $F\subset \X$ un subespacio cerrado, que será compacto por ser $\X$ compacto. Por otro lado, $f(F)$ es compacto al tratarse de la imagen continua de un compacto y, finalmente, sabemos que ser compacto en un espacio Hausdorff implica ser cerrado.
\end{proof}
\begin{obs}[Identificaciones]
	De aquí se deduce que si además $f$ es sobreyectiva, entonces se trata de una identificación (véase la proposición \ref{const_prop_identif}) y que si es biyectiva, entonces estamos ante un homeomorfismo. 
	
	Esto es muy útil para saber cuando una aplicación entre $\X$ e $\Y$ es una identificación, ya que por lo general trabajaremos con $\Y\subset\R^n$, por lo que $\Y$ será \hausdorff.
	
	En caso de ser $f$ continua y $\X$ compacto (de ahí que a veces se pidan condiciones de compacidad al dominio fundamental) se tiene que $\Y$ es un ``modelo geométrico'' del cociente $\X/\mathord{\sim}$.
\end{obs}
\begin{defi}[Variantes de la compacidad] Existen varias nociones estrechamente relacionadas con la compacidad:
	\begin{enumerate}
		\item \tbi[secuencialmente compacto]{Secuencialmente compacto}: un espacio topológico	$\X$ es secuencialmente compacto si toda sucesión en $\X$ tiene una subsucesión convergente en $\X$.
		\item \tbi[sigma-compacto@$\sigma$-compacto]{\boldmath$\sigma$-compacto}:  un espacio topológico $\X$ es $\sigma$-compacto si es la unión numerable de subespacios compactos.
		\item \tbi[numerablemente compacto]{Numerablemente compacto}: Un espacio $\X$ es numerablemente compacto si cada recubrimiento numerable tiene un subrecubrimiento finito.
		\item \tbi{Lindelöf}: véase en \ref{lindel}
	\end{enumerate}
\end{defi}

\begin{obs}[Relaciones entre las variantes]
	\label{comp_obs_relacion_defs_compacidad}
	La relación que hay entre ser compacto, secuencialmente compacto, $\sigma$-compacto, numerablemente compacto y Lindelöf es la siguiente (las comprobaciones son sencillas).
	\begin{enumerate}
		\item Compacto $\implies$ $\sigma$-compacto
		\item $\sigma$-compacto $\implies$ Lindelöf
		\item $\sigma$-compacto $\implies$ Numerablemente compacto
	\end{enumerate}
	En general, secuencialmente compacto no implica compacto ni compacto implica secuencialmente compacto. Sin embargo en un espacio métrico, las nociones de compacidad y secuencial--compacidad son equivalentes. En la literatura a este hecho se le conoce como teorema de Bolzano--Weierstrass (es una de sus implicaciones).
\end{obs}

Ahora, presentamos un resultado sobre espacios secuencialmente compactos que va a usarse con asiduidad en la sección de topología algebraica.

\begin{lem}[Lebesgue]
	\label{comp_lem_lebesgue}
	Sea $\X$ un espacio métrico compacto.
	
	Entonces, dado un recubrimiento abierto $\mc{A}$ de $\X$, existe un $\varepsilon>0$ tal que cada subconjunto de $\X$ de diámetro menor que $\varepsilon$ está contenido en algún abierto de $\mc{A}$.
	
	En particular, cada bola $\bola(x,\frac{\varepsilon}{2})$ está contenida en algún abierto de $\mc{A}$.
\end{lem}
\begin{proof}
	Como acabamos de ver en la observación \ref{comp_obs_relacion_defs_compacidad}, $\X$ es secuencialmente compacto. Es esto lo que vamos a usar para la demostración.
	
	Supongamos que existe un recubrimiento abierto $\mc{A}$ de $\X$ que no cumple la condición del lema, esto es,que para todo $\varepsilon > 0$ hay algún subconjunto de diámetro menor que $\epsilon$ que no está contenido en ningún abierto de $\mc{A}$.
	
	Entonces, vamos a construir una sucesión de ``puntos malos''. Para cada $n\in\N$, sea $A_n$ un subconjunto de diámetro menor que $\frac{1}{n}$ que no esté contenido en ningún abierto de $\mc{A}$, y cogemos un $x_n\in A_n$ como $n$--ésimo término de nuestra sucesión.
	
	Por ser $\X$ secuencialmente compacto, la sucesión $\{x_n\}_{n=1}^\infty$ tiene una subsucesión $\{x_{n_k}\}$ convergente a un punto $x\in\X$.
	
	Tomamos entonces un abierto $U\in\mc{A}$ que contenga a $x$, y un $r>0$ de forma que $\bola(x,r)\subset U$.
	
	Consideramos la bola $\bola(x,\frac{r}{2})$. Por ser $\{x_{n_k}\}$ convergente, podemos escoger un $i$ tal que $x_{n_i}\in\bola(x,\frac{r}{2})$, y además  $\frac{1}{n_i}<\frac{r}{2}$.
	
	Entonces, $x_{n_i}\in A_{n_i}\cap\bola(x,\frac{r}{2})$. como el diámetro de $A_{n_i}$ es menor que $\frac{r}{2}$, es claro que  $A_{n_i}\subset\bola(x_{n_i},r)\subset U$, lo que contradice a la hipótesis.
\end{proof}
\section{Comportamiento topológico}
En esta sección se estudiarán las distintas relaciones de la compacidad con las construcciones habituales (subespacios, cocientes, productos y sumas)

La ventaja que tenemos en este caso es que gran parte del trabajo ya está hecho. Veámoslo rápidamente.
\begin{itemize}
	\item La compacidad es hereditaria para subespacios cerrados como se vio en la proposición \ref{comp_prop_cerradoCompactoCompacto}. Esto no ocurre en general para subespacios abiertos. Para ello basta considerar $(0,1)$ como subespacio de $[0,1]$.
	
	\item Los cocientes de compactos son compactos, de hecho, como vimos en la proposición \ref{comp_prop_compContComp}, la imagen continua de un espacio compacto es compacta. Nótese que los cocientes no son más que imágenes continuas ``drásticas''.
	
	\item Las sumas finitas de espacios compactos son compactos, la demostración de esto es totalmente análoga a la del lema \ref{num_lem_lindelofSumas}.
	\item Los productos finitos de espacios compactos son compactos. Este resultado se conoce como el teorema de Tychonoff, que también es valido para el productos infinitos, aunque con una demostración distinta a la que veremos aquí. 
\end{itemize}
Señoras y señores, presentamos sin más dilación, al único, al inigualable, al que probablemente sea uno de los teoremas más importantes que veremos en estas notas. El teorema de Tychonoff.
\begin{theo}[Teorema de Tychonoff] Sean $\X$ e $\Y$ dos espacios topológicos y sea $\X \times \Y$ su espacio producto. Entonces $\X$ e $\Y$ son compactos si y solamente si $\X\times \Y$ lo es.
\end{theo}
\begin{proof}
	Demostremos ambas implicaciones a palo seco.
	\begin{itemize}[align=left, leftmargin=*]
		\item[$\bla$] Si $\X\times \Y$ es compacto, consideremos la proyección $p_{\X}: \X\times \Y \to \X$. Al ser continua y como la imagen continua de un compacto es compacto, hemos terminado. El caso para $\Y$ es análogo. 
		\item[$\bra$] Dado el producto $\X\times \Y$, podemos tomar un recubrimiento por abiertos suyo de modo que 
		\[\X\times \Y=\bigcup_{i\in J}\W_i.\]
		Nuestro objetivo es conseguir extraer una cantidad finita de esta familia de abiertos. Antes de comenzar, nótese que estos abiertos no tienen por qué ser básicos, y esto dificulta nuestra labor.
		
		En primer lugar, para todo $(x,y)\in\X\times \Y$ existe $i\in J$ de modo que $(x,y)\in \W_i$. Así, se pueden encontrar entornos abiertos $\U^y_x$ de $x$ (donde el superíndice denota que depende del $y$ tomado) y $\V_y^x$ de $y$ (que de nuevo depende del $x$ tomado) tales que
		\begin{equation*}
		(x,y)\in \U^y_x\times \V_y^x \subset \W_i
		\end{equation*}
		puesto que son base de la topología. Nótese que para otro punto del producto con el mismo $x$ y distinto $y$ los dos abiertos cambian.
		
		Fijemos $x\in \X$. De esta forma ocurrirá que $\Y=\bigcup_{y\in \Y} \V_y^x$. Por ser $\Y$ compacto, existe un subrecubrimiento finito tal que 
		\begin{equation*}
		\Y=\V^x_{y_1}\cup \dots \cup \V^x_{y_r}
		\end{equation*}
		Pese a que los puntos $y_l$ (y la cantidad de ellos, $r$) dependen de la elección de $x$, no se ha incluido en la notación con objeto de no sobrecargarla. 
		
		A continuación, para cada $x\in \X$ se puede considerar el abierto
		\begin{equation*}
		\U_x:=\U_x^{y_1}\cap \dots \cap \U_x^{y_r}
		\end{equation*}
		Consideramos el recubrimiento abierto de $\X$ formado por los $\U_x$, del cual, por ser $\X$ compacto, extraeremos un subrecubrimiento finito.
		\begin{equation*}
		\X=\bigcup_{x\in \X}\U_x\ra \X=\U_{x_1}\cup \dots \cup \U_{x_s}
		\end{equation*}
		
		A estas alturas, podemos afirmar que, dado un punto $(x,y)$, para cierto $k$ entre $1$ y $s$ y cierto $l$ entre $1$ y $r$, se verifica que
		\[(x,y)\in \U_{x_k}\times \V^{x_k}_{y_l}\subset \U^{y_l}_{x_k}\times \V^{x_k}_{y_l}\subset W_{kl}\]
		
		Esto se debe a que claramente, existe un $k\in\{1,\cdots,s\}$ tal que $x\in \U_{x_k}$. Además, tomando $x=x_k$, existe un $l\in\{1,\cdots,r\}$ tal que $y\in  \V^{x_k}_{y_l}$, luego $(x,y)\in W_{kl}$. Esto demuestra que los $W_{kl}$ son un subrecubrimiento de $\X\times \Y$.
		
		Como son una cantidad finita hemos terminado.
	\end{itemize} 
	Aunque hayamos enunciado y demostrado el teorema para dos factores, todo producto finito de compactos es compacto, esto es inmediato de ver por inducción.
\end{proof}
El teorema de Tychonoff tiene consecuencias sorprendentes en el ámbito de los espacios euclídeos usuales.
\begin{obs}[Teorema de Heine--Borel] El grandioso y todopodedoroso teorema de Heine--Borel es consecuencia directa del teorema de Tychonoff.
	\begin{enumerate}
		\item Los \tbi{adoquines} 
		$[a_1,b_1]\times \ldots \times [a_r,b_r]$
		son compactos, puesto que $[a_i,b_i]$ es compacto para todo $i\in \{1,\ldots, n\}$
		\item Un conjunto $K$ es compacto si y solo si es cerrado y acotado.
		
		Una implicación es evidente por ser $\R^n$ un espacio métrico.
		
		Recíprocamente, si está acotado existe una bola $\bola(0,\rho)$ para algún $\rho >0$ tal que $K\subset (0,\rho)$.
		
		Asimismo, esta bola está contenida en  $[-\rho,\rho]^n$, que es compacto por ser un adoquín.
		
		Además, $K$ es cerrado, luego se trataría de un cerrado contenido en un compacto, luego de un compacto.\qedhere
	\end{enumerate} 
\end{obs}
Presentamos pues, como colofón nuestra típica tabla--resumen (ver tabla \ref{Tabla_compacidad}).
\begin{table}[h!]
	\centering
	\begin{tabular}{c|c|c|c|c|}
		\cline{2-5}
		& \textbf{Subespacios}                                                 & \textbf{Cociente} & \textbf{Producto} & \textbf{Suma} \\ \hline
		\multicolumn{1}{|l|}{\textbf{Compacidad}} & \begin{tabular}[c]{@{}l@{}}Sí, en el caso\\ de cerrados\end{tabular} & Sí*               & Sí                & Sí            \\ \hline
	\end{tabular}
	\caption{Tabla resumen de compacidad.}
	\label{Tabla_compacidad}
\end{table}

Sin dar tiempo a que el lector se recupere de una demostración de un teorema gordo, vamos a ser crueles y a meter capón una observación en forma de teorema que da que pensar, aunque no nos debemos dejar engañar.
\begin{prop}[Suficiencia de la base]
	\label{comp_prop_suficienciaBase}
	$K$ es compacto si y solo si para todo recubrimiento $\mc{A}$ de abiertos de una base $\B$ hay un subrecubrimiento finito.
\end{prop}
\begin{proof}
	La implicación a la derecha es evidente, probamos pues la otra.
	
	Sea $\mc{A}$ un recubrimiento arbitrario, consideremos el recubrimiento
	\begin{equation*}
		\mc{A}_\B:=\{B_i\subset A \midc A\in\mc{A},\ B_i\in\B\}
	\end{equation*}
	$\mc{A}_\B$ es un recubrimiento básico de $K$. Por hipótesis podremos extraer un subrecubrimiento finito $B_1,\dots,B_n$.
	
	De esta forma tenemos que, como cada $B_i$ está contenido en un $A\in\mc{A}$
	\begin{equation*}
		K=\bigcup_{i=1}^nB_i\subset \bigcup_{i=1}^n A_i
	\end{equation*}
	De manera que $K$ es compacto.
\end{proof}
\begin{obs}[Tychonoff]
	Parece que la proposición \ref{comp_prop_suficienciaBase} nos hubiera venido al pelo para demostrar el teorema de Tychonoff de una patada, sin embargo, solo lo parece. Veamos esto en detalle.
	
	Si tomamos un recubrimiento básico de un espacio producto $\X\times \Y$, digamos $\{U_i\times V_i\}_{i\in J}$ y tomamos un recubrimiento finito $\{U_k\}_{k=1}^s$ y $\{V_l\}_{l=1}^r$, es claro que $\{U_k\times V_l\}$ es un recubrimiento de $\X \times \Y$, no obstante, no es un subrecubrimiento, ya que nadie nos asegura que todas las combinaciones $U_k\times V_l$ que necesitemos vayan a estar en el recubrimiento original.
\end{obs}
\begin{obs}[Simplificaciones]
	A pesar del chasco inicial de la observación anterior, es bastante claro que la proposición \ref{comp_prop_suficienciaBase} es de enorme utilidad, además de ser extrapolable a otros contextos como Lindelöf (¡compruébese!). Algunos ejemplos de posible aplicación de esta proposición son una simplificación de la intricada demostración del teorema \ref{num_teo_2axioLindel} a una demostración trivial, entre otras.
\end{obs}
\section{Compacidad local}

Hasta ahora no hemos hecho más que revisar lo que ya sabíamos de compacidad y traducirlo al nuevo lenguaje, exceptuando algún resultado como el Teorema de Tychonoff. Todo esto era de alguna manera ``global", pues se refería  a espacios topológicos. Ahora pasaremos a lo local, relacionado con entornos de puntos. Introduciremos esta noción de local con la compacidad y la extenderemos a conexión y conexión por caminos más adelante. Esto es algo novedoso y de gran importancia, por lo que se ruega al lector que lo estudie con detenimiento.

\begin{defi}[Compacidad local] Un espacio es \tbi{localmente compacto} cuando cada punto $x\in\X$ tiene una base de entornos compactos
	\[\Va{x}=\{K\subset \X: K\text{ es entorno y es compacto}\}.\]
\end{defi}

Veamos algunos ejemplos para aclarar este concepto. 

\begin{exa}[Miscelánea de ejemplos]\
	\begin{enumerate}
		\item $\R^n$ con la topología usual es localmente compacto, ya que para cada punto existe una base de entornos de la forma
		\[\Va{x}=\{\adher{\bola}(x,1/k):k\geq 1\}\]
		Dado que $\R^n$ no es compacto, de este ejemplo se desprende que, como era de esperar, la local--compacidad es más débil que la compacidad.
		\[\text{Localmente compacto }\centernot\implies \text{ Compacto}\]
		
		\item $\Q\subset\R$ con la topología usual no es localmente compacto. En efecto, tomemos un entorno compacto $0$ contenido en $\mathbb{Q}$, digamos $K_0$. Este entorno, por definición, contiene un abierto (sin pérdida de generalidad de la base) $\U$ de la topología relativa de modo que $(-\varepsilon,\varepsilon)\cap \mathbb{Q}\subset K_0$ para cierto $\varepsilon>0$.
		
		A continuación, elijamos un irracional $\theta \in (-\varepsilon,\varepsilon)$ y una sucesión $\{q_k\}_{k=1}^\infty\in (-\varepsilon,\varepsilon)\cap \mathbb{Q}$ que converge a $\theta$, lo cual siempre podemos hacer al ser $\R=\adher{\Q}$ y ser $\R$ I axioma (véase la observación \ref{num_obs_construccion}).
		
		Claramente, $\{q_k:k\in \mathbb{N}\}$ es un conjunto infinito (ya que si fuese finito convergería a alguno de ellos, por lo que el límite sería racional) contenido en $K_0$ compacto. Por el teorema de Bolzano--Weierstrass, este conjunto tiene un punto de acumulación en $K_0$ o, equivalentemente, existe una subsucesión $\{q_{k_l}\}_{l=1}^\infty$ que converge a un punto de $K_0$ al que denotaremos $p$.
		
		No obstante, toda subsucesión de una sucesión converge al mismo punto que esta última (¡compruébese!), luego hemos probado que $p=\theta$. Esto es una contradicción, pues $K_0\subset \Q$, y hemos terminado.\qedhere
	\end{enumerate}
\end{exa}

Sin embargo, parece muy engorroso tener que demostrar que existe una base de entornos con todos ellos compactos para ver que un espacio topológico es localmente compacto. Sorprendentemente, si el espacio es \hausdorff, la tarea es mucho más sencilla, no obstante, usaremos toda nuestra artillería pesada.

\begin{prop}[Local--Compacidad y \hausdorff]
	Sea $\X$ un espacio \hausdorff. Si $x\in \X$ tiene un entorno compacto, entonces tiene una base de entornos compactos.
\end{prop}
\begin{proof}
	Vamos a construir la base de compactos a partir del entorno compacto. Denotemos por $K$ a dicho entorno compacto.
	
	Se verifica que existe un abierto $U$ tal que $x\in U\subset K$. Veamos que para cada entorno abierto $W$ arbitrario existe un $L\subset W$ que sea entorno compacto de $x$.
	
	Como $K\setminus W=K\setminus (W\cap K)\subset K$ y $K\setminus (W\cap K)$ es cerrado en $K$, entonces $K\setminus W$ es compacto.
	
	Sin embargo $x\notin K\setminus W$. Vamos a aprovechar este compacto para construir uno que sí sea entorno de $x$.
	
	Como $X$ es \hausdorff, dados un compacto y un punto podemos encontrar dos entornos disjuntos (como se recalcó en el corolario \ref{comp_obs_compSep}). Entonces, podemos encontrar dos abiertos $V_x$ y $G$ tales que $V_x\cap G=\emptyset$, con $x\in V_x$ y $K\setminus W\subset G$.
	
	Podemos tomar $V_x$ además de forma que $V_x\subset U$. Si $V_x$ no cumpliera esto, tomamos en su lugar $V_x\cap U$ que también contiene a $x$. Entonces, $x\in V_x\subset \adher{V}_x\subset \adher{K}=K$, donde la última igualdad se cumple por ser $X$ Hausdorff. Entonces $\adher{V}_x$ es un compacto (por ser cerrado en compacto) entorno de $x$.
	
	Ahora, solo queda comprobar que $\adher{V}_x\subset W$. Veamos para ello que se verifica el siguiente aserto
	\[V_x\cap G=\emptyset\implies \adher{V}_x\cap G=\emptyset\]
	En efecto, si hubiera un $y\in \adher{V}_x\cap G$. Entonces $y\in\adher{V}_x$ e $y\in G$, como $G$ es abierto, será entorno de $y$, y por definición de adherencia $V_x \cap G\not=\emptyset$, llegando así a un absurdo.
	
	Ahora, como $K\setminus W\subset G$, por lo anterior $\adher{V}_x\cap (K\setminus W)=\emptyset$, y como $\adher{V}_x\subset K$, entonces es claro que $\adher{V}_x\subset W$.
\end{proof}

\section{Comportamiento topológico de la compacidad local}

Como hemos hecho en temas anteriores, estudiaremos como se comporta la compacidad local en subespacios, sumas, productos y cocientes de espacios localmente compactos.

\begin{itemize}
	\item La suma finita de espacios localmente compactos, es localmente compacta.
	
	En efecto, dado $x\in\sum_{i=1}^r \X_i$ entonces existe un $i\in\{1,\cdots,r\}$ tal que $x\in \{i\}\times \X_i$. Como $\X_i$ es localmente compacto, existe una base $\Va{x}^i$ de entornos compactos de $x$. Entonces, pasándolo por la inclusión que corresponda (continua), $\{i\}\times \Va{x}^i=\{\{i\}\times V : V\in\Va{x}^i\}$ es base de entornos compactos de $x$ en $\sum_{i=1}^r \X_i$.
	
	\item El producto finito de espacios localmente compactos es localmente compacto.
	
	Dado $x=(x_1,\dots,x_r)\in\prod_{i=1}^r\X_i$, como cada $\X_i$ es localmente compacto, exite $\Va{x}^i$ base de entornos compactos en cada uno de ellos. Entonces, 
	\[\Va{x}=\{V_1\times\cdots\times V_r: V_i\in\Va{x}^i\}\]
	es base de entornos compactos, por el teorema de Tychonoff.
	
	\item Si la identificación es abierta entonces el cociente de un espacio localmente compacto también lo es.
	
	Por ser abierta la identificación manda entornos a entornos. Por ser además continua, la imagen de compactos es compacto. Luego el cociente es localmente compacto. En el ejemplo~\ref{comp_exa_RZ} se muestra que es imprescindible que sea abierta. Si esto no ocurre, el cociente no es necesariamente localmente compacto.
	
	\item Los subespacios ``localmente cerrados'' son localmente compactos. Haremos esto con detenimiento en la sección \ref{comp_localCerrado}.
\end{itemize}

\begin{exa}[Identificación cerrada]\label{comp_exa_RZ}
	Veamos un ejemplo en el cual el cociente de un espacio topológico localmente compacto no lo es. Esto se debe a que la identificación no es abierta.
	
	Sabemos que $\R$ es localmente compacto. Sin embargo, la ya célebre rosa de infinitos pétalos $\R/\Z$ (ver ejemplo \ref{num_exa_RZ}) no lo es.
	
	Veamos que el punto $\Z$ del cociente no tiene ningún entorno compacto.
	
	Supongamos que sí. Entonces existe $K$ compacto de $\R/\Z$ tal que $\Z\in\U\subset K$, siendo $\U$ un abierto del cociente, es decir, es imagen por la proyección canónica $p$ de un abierto saturado $\W$ de $\R$ y, por tanto, $\Z\subset\W$.
	
	Entonces, para cada $k\in\Z$ podemos tomar $t_k\notin\Z$ tal que $t_k\in\W$ (pues $\W$ es abierto de $\R$ y, por tanto, es unión de intervalos abiertos). Así, $p(t_k)=t_k\in\U=p(\W)$ para todo $k\in\Z$. Consideramos entonces el conjunto de puntos aislados
	\[\{t_k : k\in\Z\}\subset\U\subset K,\]
	que es un subespacio infinito en un compacto, por lo que tiene algún punto de acumulación $a\in K$ (que no es ninguno de ellos por ser todos aislados). Entonces hay dos posibilidades: $a=\Z$ o $a\not=\Z$.
	\begin{itemize}
		\item Si $a=\Z$, se llega a una contradicción ya que  $p(\R\setminus\{t_k:k\in\Z\})$ es un entorno de $\Z$ que no corta a $\{t_k : k\in\Z\}$, por lo que $\Z$ no puede ser el punto de acumulación.
		\item Si $a\not=\Z$, y además $a\not\in\{t_k : k\in\Z\}$ se llega a un absurdo por el mismo procedimiento que en el apartado anterior. Si $a=t_{k_0}$ para cierto $k_0$ basta tomar el entorno $p(\R\setminus\{t_k:k\in\Z,\ k\not=k_0\})$.
	\end{itemize}
	Se concluye pues que el subespacio infinito $\{t_k : k\in\Z\}$ de $K$ compacto no tiene puntos de acumulación. Llegamos así a un absurdo, por lo que no existen entornos compactos de $\Z$.
	
	Nótese que la identificación es cerrada. En efecto, sea $\F$ cerrado de $\R$, recordemos que los cerrados del cociente son las imágenes de cerrados saturados de $\R$. Así, si es saturado entonces $p(\F)$ es cerrado de $\R/\Z$. Si no es saturado es porque $\emptyset\not=\F\cap\Z\not=\Z$. Entonces, $\F\cup\Z$ es cerrado saturado de $\R$ y $p(\F)=p(\F\cup\Z)$ es cerrado del cociente.
	
	Esto nos muestra que no vale con que tenga alguna propiedad igual de fuerte que ser abierta, sino que tiene que ser esta misma, no vale ninguna otra.
\end{exa}

\begin{table}[h]
	\centering
	\begin{tabular}{c|c|c|c|c|}
		\cline{2-5}
		& \textbf{Subespacios}                                                           & \textbf{Cociente}                                                                       & \textbf{Producto} & \textbf{Suma} \\ \hline
		\multicolumn{1}{|c|}{\textbf{\begin{tabular}[c]{@{}c@{}}Compacidad\\ local\end{tabular}}} & \begin{tabular}[c]{@{}c@{}}Sí, en el caso de\\ localmente cerrados\end{tabular} & \begin{tabular}[c]{@{}c@{}}Sí, en el caso de\\ identificaciones\\ abiertas\end{tabular} & Sí                & Sí            \\ \hline
	\end{tabular}
	\caption{Tabla resumen de compacidad local.}
	\label{Tabla_compacidad_local}
\end{table}

\subsection{Subespacios localmente cerrados}
\label{comp_localCerrado}
Antes de comenzar a elaborar el último apartado del estudio del comportamiento de la local compacidad conviene definir subespacio localmente cerrado.

\begin{defi}[Local--clausura]
	Un subespacio $\Y\subset\X$ es \tbi{localmente cerrado} si es abierto en su adherencia,
	\begin{equation*}
		\Y=\adher{\Y}\cap G,
	\end{equation*}
	donde $G$ es abierto de $\X$.
\end{defi}

Vayamos estudiando qué subespacios de un espacio topológico $\X$ localmente compacto lo son también. Finalmente concluiremos que la compacidad local se hereda a los subespacios localmente cerrados (y a ninguno más).

Comenzamos viendo el comportamiento de los subespacios cerrados.
\begin{lem}[Cerrados y local--compacidad]
	La local--compacidad se hereda a subespacios cerrados.
\end{lem}
\begin{proof}
	Tomemos $\F\subset\X$ subespacio cerrado. Por ser $\X$ localmente compacto para todo $x\in\F$ hay una base de entornos compactos, a los cuales denotaremos por $K$. Entonces, $\F\cap K$ es entorno de $x$ en $\F$. Además es compacto, ya que es un cerrado relativo de $K$ compacto. Por tanto, $\F$ es localmente compacto.
\end{proof}
Estudiemos ahora los subespacios abiertos.
\begin{lem}[Abiertos y local--compacidad]
	La local--compacidad se hereda a subespacios abiertos.
\end{lem}
\begin{proof}
	Consideremos ahora $G\subset \X$ subespacio abierto. Para todo $x\in G$ existe una base de entornos compactos $\Va{x}$.
	
	Entonces el conjunto $\matheuler{W}_x:=\{K\in \Va{x}\midc K\subset G\}$
	es base de entornos compactos de $x$ en $G$ (la comprobación es inmediata, basta usar el lema \ref{etop_lem_otrasProp}), siendo entonces $G$ localmente compacto.
\end{proof}
Así pues,los subespacios abiertos y cerrados de $\X$ son localmente compactos. Imaginemos ahora que tenemos la siguiente cadena de inclusiones, donde $\F$ es cerrado de $\X$ y $G$ es abierto de $\F$ (no necesariamente de $\X$)
\[G\subset\F\subset\X.\]
Por lo que acabamos de ver, si $\X$ es localmente compacto, $\F$ también lo es. Entonces $G$ es subespacio abierto de $\F$ localmento compacto, luego $G$ lo es también. En este caso además podemos escribir (por topología relativa) $G=\F\cap W$, donde $W$ es abierto de $\X$. Lo mismo ocurre con la cadena
\[\F\subset G\subset\X,\]
donde podemos escribir $\F=G\cap H$, con $H$ cerrado de $\X$.
\begin{obs}[Intersecciones mixtas]
	Si el lector ha tenido la simpatía de leer la divagación anterior se habrá dado cuenta de que si un subespacio $\Y$ de $\X$ (siendo $\X$ localmente compacto) es intersección de un abierto y un cerrado, entonces es localmente compacto. Comprobémoslo.
	
	En efecto, si $\Y=G\cap H$ con $G$ abierto y $H$ cerrado, es claro que $\Y$ es un cerrado del subespacio abierto $G$. Como $G$ es abierto y $\X$ es localmente compacto, entonces $G$ es localmente compacto y como $\Y$ es cerrado en $G$, $\Y$ es localmente compacto.
\end{obs}
Veamos finalmente qué relación guarda ser intersección de un abierto y un cerrado y ser localmente cerrado.
\begin{prop}[Caracterización de la local--clausura]
	$\Y$ es localmente cerrado si y solo si $\Y$ puede escribirse como intersección de un abierto y un cerrado de $\X$.	
\end{prop}
\begin{proof}
	Dado $\Y\subset\X$ localmente cerrado. Por definición, $\Y=\adher{\Y}\cap G$ con $G$ abierto de $\X$ e $\adher{\Y}$ cerrado de $\X$.
	
	Recíprocamente, sea $\Y=\F\cap G$ con $G$ abierto y $\F$ cerrado de $\X$. Entonces, como $\Y\subset\F$, se tiene que $\adher{\Y}\subset\F$. Además, como $\Y\subset\adher{\Y}$ e $\Y\subset G$, entonces $\Y\subset \adher{\Y}\cap G$. Así,
	\[\Y\subset \adher{\Y}\cap G\subset \F\cap G=\Y\]
	concluyéndose que $\Y=\adher{\Y}\cap G$, luego $\Y$ es localmente cerrado.
\end{proof}

Queda demostrado así que los subespacios localmente cerrados son localmente compactos. Pero ¿se dará el recíproco? La respuesta corta es no. La respuesta larga es no pero\dots y la vemos en forma de proposición.

\begin{prop}[Caracterización de la herencia]
	Si $\X$ es un espacio topológico \hausdorff e $\Y\subset\X$ es localmente compacto. Entonces $\Y$ es localmente cerrado.
\end{prop}
\begin{proof}
	Tenemos que probar que $\Y$ es abierto en su adherencia, es decir, que para todo $y\in\Y$ exite un abierto de la adherencia $G=U\cap\adher{\Y}$ (con $U$ abierto de $\X$ que contiene a $y$) tal que $y\in G\subset \Y$. Dicho de otra forma, que es entorno de todos sus puntos en la adherencia.
	
	Sea $y\in\Y$. Como $\Y$ es localmente compacto existe un entorno compacto $K\subset\Y$ de $y$. Por ser entorno, existe un abierto $W\cap\Y$ de $\Y$ tal que $y\in W\cap\Y\subset K$. Veamos que basta con tomar $U=W$.
	
	En efecto, dado que $W$ es abierto de $\X$, se tiene que $W\cap \adher{\Y}$ es abierto de la adherencia que contiene a $y$. Si probásemos que $W\cap \adher{\Y}\subset \adher{W\cap\Y}$ entonces, como $\X$ es \hausdorff, $K$ es cerrado, y esto implicaría que
	\[W\cap \Y\subset K=\adher{K}\ra \adher{W\cap\Y}\subset K\subset\Y\ra W\cap \adher{\Y}\subset\Y \]
	siendo así $W\cap \adher{\Y}$ abierto de la adherencia tal que $Y\in W\cap \adher{\Y}\subset\Y$, quedando demostrada la proposición.
	
	Probemos pues que $W\cap \adher{\Y}\subset \adher{W\cap\Y}$. Dado $z\in W\cap \adher{\Y}$, entonces $z\in \adher{\Y}$, por lo que para todo entorno abierto $V_z$ se tiene que $V_z\cap\Y\not=\emptyset$. Además, $z\in W$, así que $V_z\cap W$ es entorno de $z$. En particular para este entorno se tiene que
	\[(V_z\cap W)\cap \Y\not=\emptyset\ra V_z\cap(W\cap\Y)\not=\emptyset\]
	Esto implica que $z\in \adher{W\cap\Y}$, como queríamos.
\end{proof}

\section{Compacificación de Alexandroff}

Como hemos podido ver a lo largo de este capítulo, los espacios compactos son espacios muy manejables, con buenas propiedades. ¿No sería una gran noticia que todos los espacios con los que vamos a trabajar lo fueran? Lamentablemente esto no va a ser así, pero si que vamos a lograr en muchas ocasiones hacer de nuestro espacio un subespacio de un espacio compacto.


La idea de las compactificaciones es lograr esto (convertir espacios no compactos en subespacios de compactos) añadiendo puntos al espacio original.
Veamos en primer lugar una definición de compactificación antes de pasar a ver la que da nombre a esta sección.

\begin{defi}[Compactificación]
	Llamaremos \tbi{compactificación} de un espacio $X$ a un espacio $Y$ si $X$ es homeomorfo a un subespacio denso de $Y$, e $Y$ es compacto.
\end{defi}


La condición de que el espacio sea denso en su compactificación solo tiene como objetivo, que ésta no sea innecesariamente grande. 


Vemos un breve ejemplo para terminar de aclarar la definición.
\begin{exa}[Compactificaciones ajustadas]
	Si $X\subset K$ con $K$ compacto, entonces $\Adh_K(X)$ es compacta, ya que se trata de un cerrado en un compacto. Por ende, bastará con tomar $\Adh_K(X)$ como espacio compacto que contiene una copia homeomorfa de $X$ (no hace falta coger todo $K$).
\end{exa}

Ahora pasamos a ver el caso particular de la compactificación de Alexandroff (hay muchísimas más). En primer lugar la definiremos para luego pasar a ver una serie de observaciones que nos llevaran a demostrar que realmente verifica ser compactificación.
\begin{defi}[Compactificación de Alexandroff]
	Sea $(X,\T)$ espacio topológico localmente compacto, no compacto y Hausdorff.
	
	Tomamos un punto al que pasamos a denotar como $\infty$ tal que  $\infty\notin X$.
	
	Construimos el conjunto $X^*:=X\cup\{\infty\}$ y lo dotaremos con la topología $\T^*$ formada por
	\begin{enumerate}
		\item Los abiertos de $X$ con su topología inicial.
		\item Los conjuntos $U$ que contengan a $\infty$ tales que su complementario en $X^*$ (contenido en $X$) sea compacto.
	\end{enumerate}
\end{defi}
Antes de continuar introducimos la siguiente observación inmediata.
\begin{obs}[Condición necesaria de abertura]
	\label{comp_obs_condAbert}
	Dado un abierto $U$ que contenga a $\infty$ tenemos que $X^*\setminus U$ es compacto en $X$. Como $X$ es \hausdorff, $X^*\setminus U$ es cerrado, luego $X\setminus(X^*\setminus U)$ es abierto de $X$, por la definición de resta conjuntista $X\setminus(X^*\setminus U)=U\cap X$.
	
	En conclusión, si $U\in\T^*\setminus \T$ entonces $U\cap X$ es abierto en $X$.
\end{obs}
Hemos definido de este modo la \tbi[compactificación!de Alexandroff]{compactificación de Alexandroff} aunque no hemos probado aún que $X^*$ sea un espacio compacto ni que su topología este bien definida siquiera, pasamos a hacerlo ahora.

\begin{lem}[Buena definición]
	$\T^*$ es una topología.
\end{lem}
\begin{proof} Veamos que se verifican las tres propiedades de las topologías.
	\begin{enumerate}
		\item Evidentemente el vacío está en $\T^*$ por ser abierto de $\T$. Además $X^*$ está en la topología ya que $X^*\setminus X^*$ es el vacío, que es trivialmente compacto. 
		\item Veamos que la unión arbitraria de abiertos de $\T^*$ es un abierto.
		
		Descomponemos los abiertos de esta unión en dos grupos, los abiertos que contienen a $\infty$ y los que no. De este modo tenemos  
		$\bigcup_{U_i\in\T}U_i \cup \bigcup_{\W_j\in\T^*\setminus \T}W_j$. 
		Denotaremos al miembro de la izquierda por $U$ y al de la derecha por $W$.
		
		Evidentemente $U$ es abierto por ser unión de abiertos de $\T$.
		
		Por otra parte tenemos que $X^*\setminus W=\bigcap_{j\in J} X^*\setminus W_j$ será cerrado en $X$ (por ser intersección de compactos, y por tanto cerrados) y estará contenido en $X^*\setminus W_{j_0}$, que es compacto, luego $W$ es abierto.
		
		Por último tomamos $G:=W\cup U$. Entonces tenemos \[X^*\setminus (W\cup U)= (X^*\setminus W)\cap(X^*\setminus U)\subset X^*\setminus W\]
		Como hemos visto, $(X^*\setminus U)$ es cerrado y $(X^*\setminus W)$ compacto, que es cerrado por ser $X$ Hausdorff, por lo que la intersección de ambos será un cerrado en un compacto, y por lo tanto compacta. Así, $X\setminus G$ será compacto y por lo tanto $G$ abierto.
		
		\item Veamos que la intersección finita de abiertos es abierta, probemos esto para dos abiertos $U$ y $V$, siguiéndose el resultado por inducción.
		
		Para probarlo nos basta con observar las tres situaciones distintas que pueden darse. Si ambos abiertos son usuales trivialmente su intersección será abierto. Si por el contrario ambos abiertos son no usuales, entonces $X^*\setminus(U\cap V)=(X^*\setminus U)\cup(X^*\setminus V)$ y la unión de dos compactos es compacta. Por último, en caso de haber uno de cada (digamos $\infty\in V$), $U\cap V\subset X$, luego $U\cap V=U\cap (V\cap X)$, como por la observación \ref{comp_obs_condAbert} $(V\cap X)$ es abierto, se sigue el resultado.
	\end{enumerate}
	Con estos 3 puntos hemos demostrado que $\T^*$ es topología.	
\end{proof}

Sigamos con nuestra ardua tarea de demostrar que, efectivamente, la compactificación de Alexandroff es una compactificación.

\begin{prop}[Consistencia]
	La compactificación de Alexandroff es una compactificación.
\end{prop}
\begin{proof}
	Veamos que se cumple todo lo que debe cumplirse.
	\begin{enumerate}
		\item Es fácil comprobar que la topología relativa de $X$ como subespacio de $X^*$ es $\T$, luego $X^*$ contiene una copia homeomorfa a $X$.
		\item $X^*$ es compacto, ya que, dado un recubrimiento por abiertos $\{U_i\}_{i\in J}$ de $X^*$ es claro que habrá un $U_{i_0}$ que contenga a $\infty$, siendo entonces $X^*\setminus U_{i_0}=:K$ compacto en $X$.
		
		Por lo tanto, al ser $\{U_i\}_{i\in J}$ recubrimiento por abiertos de $K$ y éste compacto, tenemos que $K\subset U_{i_1}\cup\cdots\cup U_{i_r}$ (extraemos subrecubrimiento finito del mismo).
		
		Ahora bien, como $K\subset U_{i_1}\cup\cdots\cup U_{i_r}$ y $X^*\setminus K\subset U_{i_0}$ entonces tenemos que $X^*\subset U_{i_0}\cup  U_{i_1}\cup\cdots\cup U_{i_r}$, con lo que habríamos obtenido un subrecubrimiento finito de $X^*$.
		\item Veamos que $X\subset X^*$ es abierto denso en $X^*$. Que es abierto es trivial. Ahora, para ver que es denso, tomamos un $U$ abierto no vacio tal que $\infty\in\U$ (el otro caso es evidente).
		
		Ahora bien, por hipótesis $X^*\setminus U = K$, luego complementando $U=(X^*\setminus K)$ con $K$ compacto de $X$, por lo que $U$ tiene que cortar a $X$. En caso contrario $X=K$, pero $X$ no es compacto.\qedhere
	\end{enumerate}
\end{proof}
A modo de resumen podemos decir que la compactificación de Alexandroff es, efectivamente, una compactificación. Además, la compactificación de Alexandroff preserva de $X$ la propiedad de ser \hausdorff.
\begin{lem}[Alexandroff y Hausdorff]
	$X^*$ es un espacio Hausdorff.
\end{lem}
\begin{proof}
	 $X^*$ es \hausdorff. En efecto, como $\T$ es \hausdorff, tan solo podríamos encontrar problemas a la hora de separar $x\in X$ y $\infty$. Ahora bien, como $X$ es localmente compacto, $x$ tiene un entorno $K$ compacto. Además, $X^*\setminus K$ es abierto en $\T^*$ que contiene a $\infty$.
	 
	 Con lo que hemos encontrado dos entornos $K$ (de $x$) y $X^*\setminus K$ (de $\infty$) disjuntos.
\end{proof}
Alertamos en la siguiente observación de lo poco conveniente que resulta forzar la máquina cuando se trabaja con la compactificación de Alexandroff.
\begin{obs}[Compactificaciones degeneradas]
	La compactificación de Alexandroff está pensada para espacios $X$ no compactos. La razón de esto, aparte de porque no tiene sentido hacer compactificaciones de espacios ya compactos es que el resultado que obtenemos al ``forzar la máquina'' es feo, pues obtendríamos que $\infty$ es un punto abierto. Por ende, $X$ ya no sería denso en $X^*$, lo cual, si nos ponemos exigentes podría implicar que $X^*$ no es ni siquiera compactificación.
	
	Además, el hecho de forzar la máquina rompe una propiedad importantísima de la compactificación de Alexandroff, la unicidad (que demostramos a continuación).
\end{obs}

Antes de ver que la compactificación de Alexandroff es, en cierto sentido, única, introduzcamos una definición.
\begin{defi}[Compactificación unipuntual]
	Diremos que $Y$ es una \tbi[compactificación!unipuntual]{compactificación unipuntual} de un espacio $X$ no compacto, localmente compacto y \hausdorff si se verifica que $Y$ es \hausdorff, compacto, existe una inmersión (homeomorfismo) $f:X\to f(X)\subset Y$ con $f(X)$ denso en $Y$ y además, $Y=f(X)\cup \{\omega\}$ con $\omega\not\in f(X)$.
\end{defi}

Veamos ahora una pequeña caracterización de las  compactificaciones unipuntuales, que viene a decir que el exigir que $f(X)$ sea denso en $Y$ es redundante.

\begin{lem}[Caracterización]
	Sea $X$ un espacio \hausdorff, no compacto pero localmente compacto.
	
	$Y$ es una compactificación unipuntual de $X$ si y solo si $Y$ es un espacio \hausdorff compacto tal que $Y=f(X)\cup\{\omega\}$ con $\omega\not\in f(X)$ donde $f$ es un homeomorfismo.
\end{lem}
\begin{proof}
	Para ello habrá que probar que $f(X)$ es denso en $Y$. Si no fuera así, habría un abierto $U$ de $Y$ tal que $U\cap f(X)=\emptyset$, lo cual solo puede ocurrir se $\{\omega\}$ es abierto y, por tanto, su complementario, $f(X)$ es cerrado. Como $Y$ es compacto y \hausdorff, entonces $f(X)$ sería compacto, luego también $X$, lo cual es absurdo.
\end{proof}

Tras esto ya podemos enunciar nuestro teorema de unicidad.
\begin{theo}[Unicidad]
	Sea $X$ un espacio \hausdorff, localmente compacto y no compacto.
	
	Si $Y$ es una compactificación puntual de $X$, entonces $Y$ es homeomorfo a la compactificación de Alexandroff de $X$.
\end{theo}
\begin{proof}A lo largo de la demostración denotaremos por $f$ al homeomorfismo asociado a la compactificación $Y$. Comenzamos definiendo la aplicación
	\begin{equation*}
		\begin{array}{cc}
		h:& Y \to X^*\\
		& \omega\mapsto \infty\\
		& f(x)\mapsto x
		\end{array}\qquad
		\begin{array}{cc}
		h^{-1}:& X^* \to Y\\
		& \infty\mapsto \omega\\
		& x\mapsto f(x)
		\end{array}
	\end{equation*}
	Veamos que $h$ es homeomorfismo.
	
	Evidentemente $h$ es biyectiva. Además, si fuera continua, al ser $Y$ compacto y $X^*$ ser \hausdorff, $h$ sería una aplicación cerrada, y por lo tanto homeomorfismo. Comprobemos pues la continuidad.
	
	Tomemos $U\subset X^*$, y vayamos destripando el asunto cual gorrino en una matanza.
	\begin{itemize}
		\item Si $U\subset X$, entonces $h^{-1}(U)=f(U)$, como $f$ es abierta por ser homeomorfismo, entonces $f(U)$ es abierto de $Y$.
		\item Si $\infty\in U$, entonces, por definición $U=X^*\setminus K=X\setminus K\cup\{\infty\}$ siendo $K$ compacto.
		
		Luego $h^{-1}(U)=f(X\setminus K)\cup \{\omega\}=Y\setminus f(K)$, se tiene que $h^{-1}(U)=Y\setminus f(K)$, que es abierto, ya que $f(K)$ es compacto en $Y$, que es \hausdorff, luego $f(K)$ es cerrado.\qedhere
	\end{itemize}
\end{proof}
Luego solamente hay una compacificación puntual, que es precisamente la compactificación de Alexandroff. Este hecho es un argumento potentísimo que puede librarnos de algún que otro apuro.
\begin{cor}[Caracterización]
	Un espacio $X$ es \hausdorff y localmente compacto si y solo si es un subespacio abierto de un espacio \hausdorff compacto.
\end{cor}
\begin{proof}
	La implicación a la izquierda se da por el comportamiento topológico de \hausdorff y la local--compacidad.
	
	Recíprocamente, si $X$ es compacto el resultado es evidente, luego supongamos que no lo es, en tal caso tomamos la compactificación de Alexandroff, que verifica lo que se nos pide.
\end{proof}

Para terminar esta sección vamos a ver unos pocos ejemplos de compactificaciones.
\begin{exa}[Miscelánea]
	Veremos ahora ejemplos de compactificaciones. Los primeros tendrán como objetivo entender el mecanismo de la misma. Después, pondremos ejemplos para aclarar que, como hemos visto, dado un conjunto, sus compactificaciones puntuales son homeomorfas, pero no sucede al contrario (dada una compactificación los conjuntos de los que puede provenir no tienen porque ser homeomorfos). Por último, veremos un ejemplo de compactificación no unipuntual. 
	\begin{enumerate}
		\item Se puede comprobar haciendo uso de la llamada \tbi{proyección estereográfica} que $\R^*=\esfera^1$, y más en general, $\R^{n*}=\esfera^n\subset \R^{n+1}$.
		\item Nótese que, intuitivamente, $[0,1]\not=(0,1)^*$ ya que, $[0,1]$ tiene dos puntos más que $(0,1)$. No obstante, formalmente habría que demostrar que no hay ningún homeomorfismo entre $[0,1]$ menos un punto y $(0,1)$.
		\item Se puede demostrar que $(0,1]^*=[0,1]$ y $[0,1)^*=[0,1]$, probado esto es interesante hacer notar que estos dos subespacios no son homeomorfos entre sí (a pesar de tener la misma compactificación).
		\item Siendo $X=[0,1/2)\cup(1/2,1]$ e $Y=[0,1)$ tenemos que sus compactificaciones son la misma, $X^*=Y^*=[0,1]$, no obstante, como en el caso anterior, no son homeomorfos.
		\item Puede demostrarse que los espacios proyectivos de dimensión $n$ son compactificaciones (en general no unipuntuales) de los espacios afines.\qedhere
	\end{enumerate}
\end{exa}

