%Para este capítulo se usará la abreviatura "conex".
\chapter{Conexión}
\label{conex}

La noción de conexión es otra más de las nociones que ya conocíamos en $\R^n$ que pueden generalizarse a un espacio topológico arbitrario. Un conjunto es, intuitivamente, conexo, si está hecho ``de una sola pieza''. 

\section{Definición y propiedades}

Empezaremos viendo, por supuesto, la definición, y una serie de caracterizaciones equivalentes.

\begin{defi}[Conexión]
	$\X$ es \tbi{conexo} si no es unión de dos abiertos disjuntos no vacíos, esto es, no existen $U,V\neq\emptyset$, abiertos, tales que $U\cap V=\emptyset$ y $X=U\cup V$.
\end{defi}

\begin{obs}[Conexión en subespacios]
	De nuevo, como con compacidad, la conexión en un subconjunto no necesita una definición alternativa. Dado $\mc{Y}\subset\X$, simplemente decimos que $\mc{Y}$ es conexo si lo es entendido como espacio equipado con la topología relativa. Nótese que esta definición es equivalente a la mucho más rebuscada que dábamos en $\R^n$, que involucraba abiertos relativos.
\end{obs}

\begin{prop}[Definiciones equivalentes de conexión]
	Sea un espacio topológico $\X$. Las siguientes afirmaciones son equivalentes:
	\begin{enumerate}
		\item $\X$ es conexo.
		\item $\X$ no es unión de dos cerrados disjuntos no vacíos, esto es, no existen $F,G\neq\emptyset$, cerrados, tales que $F\cap G=\emptyset$ y $X=F\cup G$.
		\item No hay ningún conjunto abierto y cerrado no trivial, esto es, $\not\exists U$ abierto y cerrado tal que $U\neq\emptyset$ y $U\neq\X$.
	\end{enumerate}
\end{prop}

\begin{exa}
	En $\R$, los conexos son los intervalos. En efecto, si $A\subset\R$ no es un intervalo, entonces por definición $\exists a\notin A$ tal que $\inf A< a < \sup A$. Entonces, $(-\infty, a)\cap A)$ y $(a,\infty)\cap A$ son dos abiertos disjuntos no vacíos cuya unión contiene a $A$.
\end{exa}

Ahora, vamos a repasar brevemente las propiedades que ya conocíamos de conjuntos conexos.

\begin{prop}[Imagen continua]
	La imagen continua de un conexo es un conexo.
	
	\begin{proof}
		Sea $\X$ conexo, $f:\X\to\mc{Y}=f(\X)$ continua. Entonces queremos ver que $\mc{Y}$ es conexo. Para ello vamos a ver el contrarrecíproco: si $\mc{Y}$ no es conexo $\X$ tampoco.
		
		Si existiera $A\subset\mc{Y}$ abierto y cerrado no trivial (esto es, que no es ni el vacío ni el total), entonces por ser $f$ continua $f^{-1}(A)$ es también abierta y cerrada, y por ser $f$ sobreyectiva $f^{-1}(A)$ no es el vacío ni el total. Por tanto $\X$ no es conexo.
	\end{proof}
\end{prop}

% NOTA: no he nombrado el teorema como teorema del pivote porque buscado en internet sólo salía en la página de Ruiz con ese nombre. En demás libros que he visto no tiene nombre.
\begin{theo}
	\label{conex_theo_pivote}
	Sea $\{A_i\}_{i\in I}$ una familia de conexos de $\X$ y existe $a\in\bigcap_{i\in I} A_i$. Entonces $\bigcup_{i\in I} A_i$ es conexo. Ahora
	
	\begin{proof}
		Supongamos que existe $S\subset\bigcup_{i\in I} A_i$ abierto y cerrado. Vamos a ver que necesariamente es el vacío o el total. Entonces, para cada $i\in I$ $S\cap A_i\subset A_i$ y es abierto y cerrado en $A_i$. Ahora, como $A_i$ es conexo, $S\cap A_i$ tiene que ser necesariamente el vacío o el total, para cada $i\in I$. Distinguimos pues dos casos:
		\begin{enumerate}
			\item Si $S\cap A_i=\emptyset\;\forall i\in I$, entonces $S=\emptyset$ y ya está.
			\item Si $\exists i\in I$ tal que $S\cap A_i=A_i$, entonces $\exists a\in S\cap A_i$ que por hipótesis verifica que $a\in S\cap A_j$ para todo $j\in I$. Entonces ninguno de los $S\cap A_j$ es vacío, y por tanto $S\cap A_j=A_j\;\forall j\in I$, es decir, $S=\bigcup_{i\in I} A_i$. \qedhere
		\end{enumerate}
	\end{proof}
\end{theo}

Este teorema es extremadamente útil para garantizar la conexión de un sinnúmero de conjuntos, y genera un gran abanico de corolarios y consecuencias. Aquí vamos a detallar algunos que se usan con frecuencia.

\begin{cor}
	Sea $\{A_i\}_{i\in I}$ una familia de conexos que verifica que $A_i\cap A_j\neq\emptyset\;\forall i\neq j$. Entonces $\bigcup_{i\in I} A_i$ es conexo.
	
	\begin{proof}
		Sea $i_0\in I$ de forma que $A_{i_0}$ sea no vacío. Entonces $A_i\cap A_{i_0}\neq\emptyset$ por hipótesis, y podemos aplicar el teorema anterior para cada $i\in I$. Ahora, escribimos:
		\[\bigcup_{i\in I} A_i = \bigcup_{i\in I} (A_i\cup A_{i_0})\]
		y como todos comparten los elementos de $A_{i_0}$, podemos aplicar el teorema de nuevo.
	\end{proof}
\end{cor}

\begin{cor}
	Sea una cadena $\{A_i\}_{i=1}^n$ de conexos que verifica que $A_i\cap A_{i+1}\neq\emptyset$. Entonces $\bigcup_{i=1}^n A_i$ es conexo.
	
	\begin{proof}
		Por inducción, se aplica el teorema a $A_1\cup A_2$, $(A_1\cup A_2)\cup A_3$, ..., $\left(\bigcup_{i=1}^k\right)\cup A_{k+1}$.
	\end{proof}
\end{cor}

\begin{obs}
	El corolario anterior también se verifica si la sucesión de conjuntos es numerable, pero no lo vamos a probar aquí.
	% AQUÍ VIENE PROBADO http://dbfin.com/topology/munkres/chapter-3/section-23-connected-spaces/problem-2-solution/
\end{obs}

El siguiente resultado, aunque también es consecuencia del teorema \ref{conex_theo_pivote}, es más que un mero corolario y merece la categoría de teorema por sí mismo.

\begin{theo}
	Sea $A$ conexo, $B$ tal que $A\subset B\subset\adher{A}$. Entonces $B$ es conexo. En particular $\adher{A}$ es conexo.
	
	\begin{proof}
		Como podemos escribir:
		\[B=\bigcup_{b\in B\setminus A} (A\cup\{b\}) \]
		y todos comparten un punto, entonces basta probar que cada $A\cup\{b\}\subset\adher{A}$ es conexo, pues en ese caso el teorema \ref{conex_theo_pivote} nos garantiza la conexión de $B$.
		
		Supongamos que $\exists C\subset A\cup\{b\}$ abierto y cerrado y no trivial. Entonces, $C\cap A\subset A$ y es abierto y cerrado en $A$. Por tanto, $A$ es conexo, y distinguimos dos casos:
		
		\begin{itemize}
			\item Si $C\cap A=\emptyset$, entonces $C=\{b\}$ y por tanto $\{b\}$ es abierto, luego, como el entorno $\{b\}$ de sí mismo no corta con $A$, $b\notin\adher{A}$, lo cual es una contradicción.
			\item Si $C\cap A=A$, $C=A$ y por tanto $A$ es cerrado, pero $b\notin A=\adher{A}$, que de nuevo es una contradicción. \qedhere
		\end{itemize}
	\end{proof}
\end{theo}