%Para este capítulo se usará la abreviatura "num".
\chapter{Numerabilidad}
\label{num}

En matemáticas, un \tbi[axiomas de numerabilidad]{axioma de numerabilidad} es una propiedad de un cierto objeto (en nuestro caso, un espacio topológico) que afirma la existencia de un conjunto numerable con ciertas propiedades. Estas restricciones sobre el espacio pueden ser más o menos fuertes y a menudo garantizan en el espacio ciertas propiedades que hacen que se parezca a los espacios que conocemos y amamos, como $\R$. En resumen, que se verifiquen ciertos axiomas de numerabilidad hacen mucho más cómodo el trabajar con ciertos espacios.

\section{Sucesiones}

Las sucesiones eran, en $\R^n$, una herramienta fundamental para el trabajo. En un espacio topológico en general son mucho menos potentes de lo que estamos acostumbrados por falta de estructura, pero aun así son dignas de mención. Una teoría más general de la convergencia para suplir esta carencia es la teoría de la convergencia de redes, pero no vamos a detallarla aquí. 

\begin{defi}[Sucesión]
	Una \tbi{sucesión} en un espacio $\X$ es una aplicación $f:\N\to \X$. Habitualmente denotaremos una sucesión por $\{x_n\}_{n=1}^{\infty}$, es decir, el conjunto ordenado de las imágenes de la función $f$, donde $x_n$ es el ``$n$-ésimo término'' de la sucesión, o sea, $f(n)$. 
\end{defi}

El concepto más importante de las sucesiones es la convergencia. Nótese que esta definición es equivalente a la que se ve en $\R^n$ con bolas.

\begin{defi}[Convergencia]
	\label{num_defi_convergencia}
	Decimos que una sucesión $\{x_n\}_{n=1}^\infty$ de $\X$ \tbi[sucesión!convergente]{converge} a $x_0\in \X$ si para cada entorno $V_{x_0}$ de $x_0$ existe $k_0\geq 1$ tal que $x_k\in V_{x_0}$ para todo $k\geq k_0$. A menudo lo expresamos como:
	\[x_0=\lim x_n = \lim_{n\to\infty} x_n\]
\end{defi}

\begin{obs}[Bases]
	Para que una sucesión sea convergente, es suficiente que la definición \ref{num_defi_convergencia} se verifique para una base de entornos. No es necesario comprobar que se cumple para un entorno cualquiera.
\end{obs}

\begin{obs}[Consecuencias de la definición de convergencia] \
	\label{num_obs_consecuencias_continuidad}
	\begin{enumerate}
		\item Si $x_0 = \lim x_k$, con $x_k\in A$, entonces $x_0\in\adher{A}$. En efecto, en todo entorno de $x_0$ hay por definición de convergencia algún punto de $A$, y entonces es punto adherente por definición.
		
		\item El límite de una sucesión no es necesariamente único, es decir, una sucesión puede converger a varios puntos. En efecto, consideramos un conjunto $\X$ con la topología del punto\indext{del punto} $\T_a$, para un punto $a\in\X$. Aquí, la sucesión constantemente $a$ converge simultáneamente a todos los puntos del espacio.
		
		Esto es claro puesto que cualquier entorno de cualquier punto contiene a $a$.\qedhere
	\end{enumerate}
\end{obs}
Para afianzar en el lector la idea de que la intuición que pudiera tener ya nos es válida aquí vamos a introducir un ejemplo.
\begin{defi}[Complementarios]
	Sea $\X$ un conjunto arbitrario, definimos la \tbitop[$\T_{\text{CN}}$]{de los complementarios numerables} como
	\begin{equation*}
		\T_{\text{CN}}:=\{\U\subset \X\midc \X\setminus\U\text{ es numerable}\}
	\end{equation*}
	De manera análoga se define la \tbitop[$\T_{\text{CF}}$]{de los complementarios finitos} $\T_{\text{CF}}$.
	
	Es un sencillo ejercicio comprobar que ambas son efectivamente topologías.
\end{defi}
\begin{lem}[Convergencia en $\T_{\text{CN}}$]
	Sea $\X$ un espacio topológico con la topología de los complementarios numerables. Entonces, una sucesión en $\X$ converge si y solo si todos sus términos son iguales a partir de cierto $k_0\in\N$.
\end{lem}
\begin{proof}Procedemos por doble implicación.
	\begin{enumerate}
		\item[$\bla$] 
		Evidentemente, si la sucesión es constantemente $x_0$ a partir de cierto término $k_0$, la sucesión converge a $x_0$ ya que todo entorno de $x_0$ contiene a $x_0$ y la sucesión es siempre $x_0$ a partir de término $k_0$, luego se cumple algo más fuerte de lo que nos pide la convergencia.
		\item[$\bra$]
		Si una sucesión converge a un punto $x_0$, el conjunto $A$ de los términos de la sucesión que no coinciden con $x_0$ es numerable (ya que el conjunto de términos de la sucesión es numerable en sí mismo). Luego su complementario, que contiene a $x_0$ es abierto, y, por tanto, entorno de $x_0$. Como $x_0$ es el límite habrá un cierto $k_0$ a partir del cual los términos de la sucesión queden confinados en $\X\setminus A$, no obstante, el único valor que puede tomar la sucesión en este conjunto es $x_0$.\qedhere
	\end{enumerate}
\end{proof}
Para ver la potencia del resultado anterior hagamos una pequeña observación.
\begin{obs}[Sucesión armónica]
	Es harto conocido que la sucesión $\{\frac{1}{n}\}_{n=1}^{\infty}$ es convergente a $0$ en $\R$ con la topología usual, no obstante no lo será, por el lema anterior en $\R$ con la topología de los complementarios numerables. Esto es evidente ya que la sucesión jamás toma el valor $0$.
\end{obs}
\section{Primer axioma de numerabilidad}

\begin{defi}[Primer axioma]
	Decimos que un espacio $\X$ es \tbi[axiomas de numerabilidad!primer axioma de numerabilidad]{primer axioma de numerabilidad} o simplemente \tb{\ti{I axioma}} si todo punto de $\X$ tiene una base de entornos numerable.
\end{defi}

\begin{obs}[Bases encajadas]
	Esta base numerable, si existe, siempre se podrá tomar encajada, como ya mencionamos en la observación \ref{etop_obs_base_encajada}. 
\end{obs}

Veamos que cuando le vamos exigiendo algunas cosas más al espacio vamos desbloqueando argumentos que usábamos con naturalidad en el contexto de los espacios métricos, haciendo de las sucesiones una herramienta cada vez más y más potente.
\begin{obs}[Construcción de sucesiones convergentes]
	\label{num_obs_construccion}
	En un espacio $\X$ I axioma podemos construir sucesiones convergentes a cualquier punto $x_0$ del mismo. El procedimiento que veremos a continuación (con ligeras modificaciones quizá) puede resultarnos muy útil a la hora de elaborar argumentos como veremos en el lema \ref{num_obs_adherencia_limites}. Veamos cómo se hace.
	
	Tomamos el punto $x_0$ al que queremos que nuestra sucesión converja. Asimismo tomamos una base numerable encajada $\{V_k\midc k\in\N\}$ del mismo. Y para cada $k\in\N$ tomamos un punto en $x_k\in V_k$.
	
	Veamos que la sucesión $\{x_k\}_{k=1}^{\infty}$ converge a $x_0$. En efecto, dado un entorno de $x_0$, siempre podemos encontrar uno de la base $V_{k_0}$ contenido. Luego todos los términos de la sucesión a partir de $k_0$ se encuentran en el entorno inicial (por ser la base encajada).
\end{obs}
\begin{lem}[Caracterización de la adherencia]
	\label{num_obs_adherencia_limites}	
	Si $\X$ es I axioma y $A$ un subconjunto de $X$. Entonces $x$ es adherente a $A$ si y solo si hay una sucesión contenida en $A$ que converge a $x$.
\end{lem}
\begin{proof}
	Vamos a ver solo la implicación hacia la derecha, la otra ya la vimos en la observación \ref{num_obs_consecuencias_continuidad}.
	
	Basta con construir una sucesión convergente a $x$ por el procedimiento de la observación \ref{num_obs_construccion} pero a la hora de seleccionar los puntos de la sucesión exigimos que además de estar en el $V_k$ correspondiente de la base encajada también estén en $A$, cosa siempre posible por ser $x$ un punto adherente a $A$.
\end{proof}

Para ver las cosas con un poco de perspectiva vamos a demostrar desde un punto de vista puramente ``secuencial'' que en la topología del punto, el punto en cuestión es denso.
\begin{exa}[Adherencias curiosas]
	\label{num_exa_adher_topologia_punto}
	Sea $\X$ un conjunto equipado con la topología $\T_a$ del punto $a\in\X$.
	
	Como la sucesión constantemente $a$ (y por tanto contenida en $\{a\}$) converge simultáneamente a todo el espacio, como vimos en la observación \ref{num_obs_adherencia_limites}, resulta que, por la observación \ref{num_obs_adherencia_limites}, el punto $a$ es denso en $\X$.
\end{exa}

El hecho de que el límite no sea único puede resultar perturbador para las mentes acostumbradas a los espacios métricos. Por esa razón a veces es buenos exigirle ciertas condiciones adicionales de separación a nuestro espacio.
\begin{prop}[Unicidad del límite]
	Si $X$ es \hausdorff, entonces el límite es único.
\end{prop}
\begin{proof}
	Si no lo fuera, podríamos tomar dos entornos disjuntos alrededor de los dos límites, y la sucesión tendría que estar en ambos al mismo tiempo a partir de un cierto punto.
\end{proof}

\begin{obs}[Espacios métricos]
	Un espacio métrico es siempre I axioma (basta tomar las bolas de radio $\frac{1}{n}$) y \hausdorff, de forma que en él las sucesiones se comportan como esperamos.
\end{obs}
Otra virtud intrínseca de los espacios primer axioma es que, en ellos, la continuidad y la convergencia de sucesiones guardan una relación muy estrecha.
\begin{prop}[Caracterización de la continuidad]
	Sea una aplicación $f:\X\to\Y$, siendo $\X$ un espacio I axioma. En estas condiciones $f$ es continua en $x_0$ si y solo si para toda sucesión $\{x_n\}_{n=1}^\infty$ convergente a $x_0$ la sucesión de los términos imagen $\{f(x_n)\}_{n=1}^\infty$ es convergente a $f(x_0)$.
\end{prop}
\begin{proof}
	Hagamos ambas implicaciones.
	\begin{enumerate}
		\item[$\bra$] Sea $V$ un entorno arbitrario de $f(x_0)$, como $f$ es continua, transforma entornos en entornos por imágenes inversas, luego $f^{-1}(V)$ es entorno de $x_0$. Por ser $x_0$ el límite, es claro que a partir de cierto $k_0\in\N$ todos los términos de la sucesión caen en $f^{-1}(V)$.
		
		Por ende, todos los términos imagen  $f(x_n)$ con $n\geq k_0$ caen en $V$, y esa es exactamente la definición de convergencia para $f(x_0)$ con la sucesión de términos imagen.
		\item[$\bla$] Recíprocamente razonamos por reducción al absurdo. Si $f$ no fuera continua en $x_0 $ habría un $V$ entorno de $f(x_0)$ tal que la imagen de cualquier entorno $W$ de $x_0$ no estuviera contenida en $V$. No obstante (y aquí es donde ponemos a funcionar el primer axioma), construyendo una sucesión $\{x_n\}_{n=1}^\infty$ convergente a $x_0$ con el método de la observación \ref{num_obs_construccion}, exigiendo además que la imagen del punto que se escoja como término no caiga en $V$ (lo cual podemos hacer por hipótesis). Tenemos que, por hipótesis $\{f(x_n)\}_{n=1}^{\infty}$ converge a $f(x_0)$, no obstante, ningún término de la sucesión está en $V$, lo cual contradice que $f(x_0)$ sea el límite.\qedhere
	\end{enumerate}
\end{proof}
\section{Otros axiomas de numerabilidad}
Exploremos otros axiomas de numerabilidad, que, a pesar de tener mucho interés, explotaremos menos asiduidad en este texto.
\begin{defi}[Segundo axioma]
	Decimos que un espacio $\X$ es \tbi[axiomas de numerabilidad!segundo axioma de numerabilidad]{segundo axioma de numerabilidad} o simplemente \tb{\ti{II axioma}} si tiene una base de abiertos numerable.
\end{defi}

\begin{defi}[Separable]
	Decimos que un espacio $\X$ es \tbi{separable} si en él existe un conjunto denso numerable.
\end{defi}

\begin{exa}[Espacios separables]
	Refrescamos algunos ejemplos de espacios separables:
	\begin{enumerate}
		\item $\R$ con la topología usual es separable, pues $\Q$ es denso y numerable. De la misma forma, $\R^n$ también lo es, por $\Q^n$. 
		
		\item Un espacio $(\X,\T_a)$, donde $\T_a$ es la topología del punto, es automáticamente separable. En efecto, ya vimos en el ejemplo \ref{num_exa_adher_topologia_punto} que la adherencia del conjunto $\{a\}$ es el total, luego es denso; y desde luego es numerable. \qedhere
	\end{enumerate}
\end{exa}

\begin{defi}[Lindelöf]
	\label{lindel}
	Decimos que un espacio $\X$ es \tbi{Lindelöf} si para todo recubrimiento por abiertos de $\X$ podemos extraer un subrecubrimiento numerable.
\end{defi}
\begin{defi}[Fuertemente Lindelöf]
	Un espacio topológico $\X$ se dice \tbi[Lindelöf!fuertemente]{fuertemente Lindelöf} si es Lindelöf y además todos sus subespacios (equipados con la topología relativa) también lo son.
\end{defi}

\begin{obs}
	Nótese que la noción de compacidad (que veremos en el capítulo \ref{comp}) es más fuerte que la de Lindelöf. De hecho, aunque no la hemos mencionado aquí, esta propiedad está a caballo entre los axiomas de numerabilidad y los de compacidad (que también veremos en el capítulo \ref{comp}).
\end{obs}

\subsection{Relaciones entre los axiomas de numerabilidad}
Las relaciones entre los axiomas de numerabilidad se resumen en el siguiente diagrama.
\begin{equation*}
	\xymatrix{
		& \text{II axioma}\ar@{=>}[dl]\ar@{=>}[d]\ar@{=>}[dr]&\\
		\text{I axioma} & \text{Lindelöf}\ar@{=>}@/^/[u]^{\star} & \text{Separable}\ar@{=>}@/_/[ul]_{\star}
	}
\end{equation*}
Donde $\star$ significa que dicha implicación únicamente se cumple para espacios métricos.

En este apartado (cuya lectura puede omitirse una primera vez) veremos las demostraciones de algunas de las implicaciones y algunos contraejemplos para las que no se dan. Se anima al lector a que intente probar algunas por su cuenta.

No obstante, para que este no malgaste su tiempo en vano, marcamos con (\Lightning)\ las demostraciones que consideramos de mayor dificultad.
\begin{obs}[Concimiento previo]
	Por las observaciones \ref{etop_2_axioma_num} y \ref{etop_2_axioma_sep} sabemos que ser II axioma es más fuerte que ser I axioma y que ser separable. 
\end{obs}
\begin{theo}[\Lightning\ II axioma $\ra$ Lindelöf]
	Todo espacio topológico II axioma es Lindelöf.
\end{theo}
\begin{proof}
	En efecto, sea $\X$ un espacio II axioma y sea $\Gamma$ un recubrimiento abierto de $\X$. Veamos que podemos extraer uno numerable.
	
	En efecto, por ser $\X$ segundo axioma, tiene una base numerable $\B:=\{B_n\in\T\midc n\in\N\}$.
	
	Para cada $n$ tomamos un abierto $\U_n$ de $\Gamma$ que verifique que $B_n\subset \U_n$ (si existe). De este modo son hemos quedado con un subrecubrimiento numerable (a lo sumo) de $\Gamma$. Para terminar tendremos que ver que, en efecto, cubre.
	
	Si hubiera algún $x\in\X$ que no viviera en ningún $\U_n$, $x$ viviría en algún $\U_i$ que no estuviera en el subrecubrimiento, sin embargo, $\U_i=\bigcup_{n\in A\subset\N} B_n$, y, por tanto, $\U_i$ contiene a algún conjunto de la base (en particular a alguno, digamos $B_{n_0}$ que contenga a $x$) luego tenemos garantizada la existencia de un $\U_{n_0}$ tal que  $x\in\B_{n_0}\subset\U_{n_0}$ que está en el subrecubrimiento. Lo que entra en contradicción con que $x$ no esté cubierto.
\end{proof}
\begin{lem}[Fuertemente Lindelöf y II axioma]
	Si $\X$ es II axioma entonces es fuertemente Lindelöf.
\end{lem}
\begin{proof}
	En efecto, si $\X$ es II axioma, $\X$ es Lindelöf, además, por la observación \ref{etop_obs_axiomasHereditarios} todos sus subespacios son II axioma (luego Lindelöf).
\end{proof}
\begin{obs}[I axioma $\protect\centernot\ra$ II axioma]
	Un contraejemplo para esto es la topología discreta en un conjunto $\X$ no numerable. Esto se debe a que cada punto es abierto.
	
	En efecto, cada punto debe poder expresarse como unión de abiertos de la base, luego la base debe contener a los propios puntos como abiertos (al menos), de esta forma debe ser forzosamente no numerable (como poco).
\end{obs}
\begin{exa}[$(\R,\T_{[,)})$ no es II axioma]
	\label{num_sorgenfrey_IIax}
	Sea $\B$ una base. Para cada $x\in\R$ es claro que $[x,\infty)$ es abierto, luego es unión de abiertos de la base. En particular, habrá un abierto de la base, al que llamaremos $B_x$ tal que $x\in B_x$. Además $B_x$ tiene a $x$ por mínimo.
	
	Por ende, hemos establecido aplicación inyectiva entre $\R$ y los conjuntos de la base, luego la base debe ser al menos no numerable.
\end{exa}
\begin{obs}[Separable $\protect\centernot\ra$ II axioma]
	Se sigue de los ejemplos \ref{num_sorgenfrey_IIax} y \ref{etop_sorgenfrey}
\end{obs}
\begin{obs}[$(\R,\T_u)$ es fuertemente Lindelöf]
	\label{num_exa_lindelof_R}
	Evidentemente, $\R$ es II axioma, tal y como se vio en el ejemplo \ref{etop_bases}, luego $\R$ es lindelöf y fuertemente Lindelöf (por ser II axioma).
\end{obs}
\begin{theo}[\Lightning\ Lindelöf $\protect\centernot\ra$ II axioma]
	La recta de Sorgenfrey es Lindelöf.
\end{theo}
\begin{proof}
	Sea $\Gamma$ un recubrimiento de abiertos Sorgenfrey de $\R$.
	\begin{equation*}
		\Gamma:=\{\U_i\in\T_{[,)}\midc i\in I\}=\left\{\bigcup_{j\in J}[x_j^i,y_j^i)\midc i\in I,\ j\in J\right\}
	\end{equation*}
	Consideramos el conjunto $\Gamma_u:=\left\{\bigcup_{j\in J}(x_j^i,y_j^i)\midc i\in I,\ j\in J\right\}$ (el resultante de quitar los extremos izquierdos).
	
	Notemos que $\Gamma_u$ puede que no sea recubrimiento de $\R$ (podría fallar algún extremo $x_j^i$). Llamemos $\U$ a la unión de todos los conjuntos de $\Gamma_u$. Como $\Gamma_u$ es un recubrimiento de abiertos usuales y $\U$ es Lindelöf por el ejemplo \ref{num_exa_lindelof_R}, podemos extraer un subrecubrimiento numerable $\Gamma_u^n$.
	
	Como $\R\setminus \U$ es un conjunto compuesto de extremos izquierdos de los ``intervalitos'' que conforman los conjuntos de $\Gamma$, si fuera numerable, bastaría con considerar el subrecubrimiento 
	\begin{equation*}
		\left\{\bigcup_{j\in J}[x_j^n,y_j^n)\midc n\in \N,\ j\in J\right\}\cup\left\{\bigcup_{j\in J}[x_j^m,y_j^m)\midc \forall m\in\N\ \exists\ j_0\in J\text{ tal que } x_{j_0}^m\in\R\setminus\U\right\}
	\end{equation*}
	Donde el miembro de la izquierda es $\Gamma_u^n$ añadiéndole el extremo izquierdo a cada intervalo y el miembro de la derecha el conjunto de los abiertos Sorgenfrey originales para los que falla alguno de los extremos izquierdos de los intervalos que lo componen.
	
	Es claro que todo esto en su conjunto es numerable y que es un subrecubrimiento de $\Gamma$.
	
	Comprobemos que $\R\setminus\U$ es numerable definiendo una aplicación inyectiva desde $\R\setminus\U$ hasta $\Q$.
	
	Sea $x\in\R\setminus\U$, luego $x$ es el extremo izquierdo de algún intervalo $(x_j^i, y_j^i)$, por densidad de $\Q$, en dicho intervalo hay algún racional $q_x$, que tomaremos como imagen de $x$.
	
	Nuestra aplicación es inyectiva, ya que si cogemos dos puntos $x,x'\in\R\setminus\U$ tomando $x<x'$ tendríamos que si $q_{x'}=q_x$ entonces $x'\in(x,y_j^i)$, lo cual es absurdo, ya que entonces $x'$ quedaría cubierto por $\Gamma_u$ y por tanto viviría en $\U$.
\end{proof}
Veamos que además ninguna de las otras posibles implicaciones se da, para ello introduzcamos el siguiente lema.
\begin{lem}[I axioma]
	Si $\X$ es no numerable, $(\X,\T_{\mathrm{CF}})$ no es I axioma.
\end{lem}
\begin{proof}
	Comprobemos que para cada punto $a\in \X$ toda base de entornos (sin pérdida de generalidad abiertos) es no numerable. Si fuera numerable, digamos 
	\[\{V_k \in\T_{\text{CF}}\midc k \geq 1\}\]
	La intersección de todos ellos es no vacía (de hecho es no numerable). En efecto, basta tomar complementarios
	\[\mathcal{X} \backslash \left(\bigcap_{k\geq 1} V_k\right)= \left(\bigcup_{k\geq 1} \mathcal{X} \backslash V_k\right)\]
	Nótese que el miembro de la derecha es numerable por ser unión numerable de conjuntos finitos. Al ser $\mathcal{X}$ no numerable y 
	\[\mathcal{X}= \left(\bigcap_{k\geq 1} V_k\right) \cup \left(\bigcup_{k\geq 1} \mathcal{X}\backslash V_k\right),\]
	la intersección de todos los abiertos de la base es no numerable.
	
	Tomemos ahora un punto cualquiera $b$ de esta intersección con la condición de que sea distinto de $a$ y consideremos el entorno abierto de $a$ dado por $W:=\mathcal{X}\backslash \{b\}$.
	
	De forma evidente la condición $V_k \subset W$ no se verifica para ningún $k$ ya que $b\in V_k$ para todo $k$. Lo que contradice que el conjunto inicial fuera base.
\end{proof}
\begin{obs}[Lindelöf $\protect\centernot\ra$ I axioma]
	En efecto, $(\X,\T_{\text{CF}})$ con $\X$ no numerable es Lindelöf ya que dado un recubrimiento arbitrario por abiertos $\Gamma$, si me quedo con uno de ellos, me queda una cantidad finita de puntos por cubrir.
	
	Como $\Gamma$ es recubrimiento hará falta una cantidad finita de abiertos de $\Gamma$ para cubrir el resto, luego puedo extraer así el recubrimiento numerable (de hecho finito).
\end{obs}

\begin{obs}[Separable $\protect\centernot\ra$ I axioma]
	Si $\X$ es no numerable $(\X,\T_{\text{CF}})$ es separable. Es más, todo conjunto abierto es denso. En efecto, si hubiera un conjunto $A\subset \X$ abierto no denso, entonces habría un abierto $\U\in \T_{\text{CF}}$ tal que $\U\cap A = \emptyset$. Tomando complementarios, se tiene
	
	\[(\X\setminus \U)\cup (\X\setminus A)=\X\]
	
	Como el miembro de la izquierda es finito y el de la derecha infinito hemos terminado.
\end{obs}

Se pueden encontrar contraejemplos para el resto de implicaciones con la topología discreta sobre diferentes conjuntos.

Pasemos a dar la idea de las demostraciones (dejando como ejercicio los detalles) de las implicaciones que únicamente se dan para espacios métricos.
\begin{lem}[Separabilidad]
	Sea $(M,d)$ espacio métrico. Si es separable, entonces es II axioma.
\end{lem}
\begin{proof}
	Sea $A$ el conjunto numerable denso. Para cada punto de $A$ tomamos el conjunto de todas las bolas de radio racional. Por ende, el conjunto $\B$ formado por todas las bolas de centro un punto de $A$ y radio racional es un conjunto numerable, siendo sencillo demostrar que es, además, una base (gracias a la propiedad triangular).
\end{proof}
\begin{lem}[Lindelöf]
	Sea $(M,d)$ espacio métrico. Si es Lindelöf, entonces es II axioma.
\end{lem}
\begin{proof}
	Para cada $n\in\N$ consideramos el recubrimiento abierto
	\begin{equation*}
		\Gamma:=\left\{\bola\left(x,\frac{1}{n}\right)\midc x\in M\right\}
	\end{equation*}
	como $M$ es Lindelöf, podemos extraer un subrecubrimiento numerable $\Gamma_n$. Si consideramos el conjunto numerable de abiertos
	\begin{equation*}
		\B:=\bigcup_{n=1}^{\infty}\Gamma_n
	\end{equation*}
	es sencillo demostrar que $\B$ es base. 
	\end{proof}
\section{Comportamiento topológico}
\label{num_comportamiento}
Estudiemos ahora el comportamiento topológico de los axiomas de numerabilidad.
\subsection{Subespacios}
Llegados a este punto, podemos recoger los frutos de nuestra cosecha, con sudor y sangre plantada, a costa de pasar hambre durante el largo y frío invierno que fue el capítulo \ref{etop}.
\begin{obs}[Conocimiento previo]
	Las observación \ref{etop_obs_axiomasHereditarios} nos dice que tanto I axioma como II axioma se heredan a subespacios.
\end{obs}
Lamentablemente, para la separabilidad el comportamiento no es tan bueno.
\begin{exa}[Topología del punto]
	Sea $(\X,\T_a)$ con $\X$ no numerable. Sabemos de sobra que el punto $a$ es denso en $\X$. No obstante, si tomamos el subespacio $\X\setminus\{a\}$, la topología relativa asociada a dicho subespacio resulta ser la discreta. Como $\X\setminus\{a\}$ es no numerable, evidentemente es no separable, ya que todo conjunto numerable es cerrado y por tanto coincide con su adherencia.
\end{exa}
A pesar de no tener la separabilidad un buen comportamiento en general, si que hay un oasis en medio del desierto, los abiertos.
\begin{lem}[Separabilidad y abiertos]
	Si $\X$ es un espacio separable, todo subespacio abierto suyo también lo es.
\end{lem}
\begin{proof}
	Sea $Y$ un subespacio abierto de $\X$. Por el lema \ref{etop_lem_otrasProp} sabemos que todo abierto relativo de $Y$ es abierto de $\X$. Como $\X$ es separable, habrá un conjunto $A$ numerable tal que todo abierto de $\X$ (y en particular los de $Y$) contiene algún punto de $A$. Por ende, $Y$ es separable. 
\end{proof}
Veamos pues como se comporta Lindelöf, que adelantamos que en general también mal.
\begin{exa}[Topología pseudodiscreta]
	Sea $\X$ un conjunto no numerable y sea $\{x\}$ tal que $x\not\in\X$. Consideramos pues el conjunto $Y:=\X\cup\{x\}$ al que equipamos con la siguiente topología.
	\begin{equation*}
		\T:=\{\emptyset, Y\}\cup\partes(\X)
	\end{equation*}
	Es claro que $Y$ es Lindelöf ya que todo recubrimiento por abiertos debe contener a $Y$, luego de todo recubrimiento podemos extraer un recubrimiento finito (el propio $Y$) luego numerable. No, obstante, $\X$ como subespacio de $Y$ tiene la topología discreta, y por tanto no es Lindelöf.
\end{exa}
Tal y como ocurría en el caso de la separabilidad, hay luz al final del túnel y la propiedad de Lindelöf si que es hereditaria para subespacios cerrados.
\begin{lem}[Lindelöf y cerrados]
	Si $\X$ es un espacio Lindelöf, todo subespacio cerrado suyo también lo es.
\end{lem}
\begin{proof}
	Sea $Y$ un subespacio cerrado Sea $\Gamma$ un recubrimiento abierto de $Y$. De esta forma podemos descomponer $\X$ de la siguiente forma
	\begin{equation*}
		\X=(\X\setminus Y)\cup\bigcup\Gamma
	\end{equation*}
	Como $\X\setminus Y$ es abierto, $\Gamma\cup(\X\setminus Y)$ es un recubrimiento abierto de $\X$. Como $\X$ es Lindelöf podemos extraer un recubrimiento numerable $\Gamma_n$, sin pérdida de generalidad podemos suponer que nos quedamos con el abierto $\X\setminus Y$. Dicho de una forma más clara
	\begin{equation*}
		\X=\X\setminus Y\cup \bigcup\Gamma_n
	\end{equation*}
	Como $Y\cap(\X\setminus Y)=\emptyset$ es claro que $\Gamma_n$ es un subrecubrimiento numerable de $Y$.
\end{proof}
\subsection{Cocientes}
El estudio de los cocientes para los casos de I y II axioma se basa en encontrar contraejemplos que muestran que pegar puntos no es buena idea si deseamos preservar estas propiedades. En concreto, nos bastará con encontrar un espacio topológico II axioma y un cociente suyo que no sea ni siquiera primer axioma.

Para encontrarlo, podemos hacer un pacto con Belcebú, quien a cambio de una demostración un poco dolorosa nos dará una flor con infinitos pétalos que abrirá la caja de Pandora.
\begin{exa}[Flor infinita]
	Consideremos el espacio topológico $\R$ con la topología usual, que es un espacio II axioma. Ahora tomemos su cociente $\R/\Z$, es decir, el resultante de relacionar a todos los números enteros entre sí y a los demás solo consigo mismo.
	
	Efectivamente, podemos imaginarnos esto como una flor con infinitos pétalos cuyo centro es el punto $\Z$ (que recordemos que es un punto en el cociente). Para demostrar que $\R/\Z$ no es primer axioma deberemos encontrar un punto que no posea una base de entornos numerable. Dicho punto, como se ve venir, es $\Z$.
	
	Para demostrar esto tomemos una familia numerable de entornos (sin pérdida de generalidad abiertos) de $\Z$ a cuyos elementos los llamaremos $U_n$. Vamos a construir un abierto $V$ que no contenga a ningún abierto de la familia. Como sabemos, los abiertos del cociente son las proyecciones de los abiertos saturados del espacio original.
	
	Para cada abierto $U_n$ y cada número entero $k$ definimos un intervalo $(k-\varepsilon_n^k,k+\varepsilon_n^k)$ de manera que se verifique
	\begin{equation*}
		p\left(\bigcup_{k\in\Z}(k-\varepsilon_n^k,k+\varepsilon_n^k)\right)\subset U_n
	\end{equation*}
	La pieza maestra de la demostración consiste en definir un $\delta_k:=\frac{1}{2}\varepsilon_k^k$, a partir del cual definimos nuestro abierto $V$.
	\begin{equation*}
		V:=p\left(\bigcup_{k\in\Z}(k-\delta_k,k+\delta_k)\right)
	\end{equation*}
	Veamos que $V$ no contiene a ningún $U_n$. En efecto, como $\delta_n<\varepsilon_n^n$ podemos considerar un elemento $x$ del intervalo $x\in(n-\delta_n,n+\varepsilon_n^n)$ que sea mayor que $n+\delta_n$. De esta forma $p(x)\in U_n$ pero no a $V$.
	
	Esto demuestra que $\R/\Z$ no es I axioma.
\end{exa}
Al contrario de lo que pasaba en el caso de los subespacios, aquí la separabilidad y Lindelöf se comportan extraordinariamente bien.

Antes de nada hay que refrescar que un espacio cociente no es más que un espacio que es resultado de transformar continuamente el espacio original, teniendo el espacio de llegada una topología un poco drástica. Aquí veremos que el cómo sea la topología del espacio de llegada nos da igual a la hora de que se preserven las propiedades de separabilidad y Lindelöf. Únicamente nos interesará que la transformación sea continua.
\begin{prop}[Imagen continua de Separable]
	Sea $\X$ un espacio separable y $f:\X\to Y$ una función continua, entonces $f(\X)$ es un espacio separable.
\end{prop}
\begin{proof}
	Sea $A$ el conjunto numerable denso en $\X$, como $f$ es continua se verifica que
	\begin{equation*}
		Y=f(X)=f(\adher{A})\subset\adher{f(A)}
	\end{equation*}
	Y como $f(A)$ es numerable hemos terminado.
\end{proof}
\begin{prop}[Imagen continua de Lindelöf]
	Sea $\X$ un espacio Lindelöf y $f:\X\to Y$ una función continua, entonces $f(\X)$ es un espacio Lindelöf.
\end{prop}
\begin{proof}
	Sea $\Gamma:=\{\U_i\in\T_Y\midc i\in Y\}$ un recubrimiento por abiertos de $Y$. Como $f$ es continua $f^{-1}(\U_i)$ es abierto en $\X$. Por ende, se verifica que $\X = f^{-1}(\bigcup_{i\in I}\U_i)=\bigcup_{i\in I} f^{-1}(\U_i)$, luego $f^{-1}(\Gamma)$ es un recubrimiento por abiertos de $\X$. Como $\X$ es Lindelöf, podemos extraer un subrecubrimiento numerable. Es decir
	\begin{equation*}
		\X=\bigcup_{n=1}^{\infty}f^{-1}(\U_n)
	\end{equation*}
	Tomando imágenes a ambos lados tenemos que $f(\X)\stackrel{!}{=}\bigcup_{n=1}^{\infty}\U_n$. Con lo que hemos extraído un subrecubrimiento numerable de $f(\X)$.
\end{proof}
\subsection{Productos}
A la hora de estudiar los productos también podemos recurrir a trabajo anterior, en este caso al capítulo \ref{const}.
\begin{obs}[Conocimiento previo]
	Sabemos por las proposiciones \ref{const_prop_base} y \ref{const_prop_base2} que el producto de espacios tanto I como II axioma es un espacio I o II axioma respectivamente.
\end{obs}
La separabilidad también es agradecida en este contexto.
\begin{lem}[Productos y separabilidad]
	Sean $X_1,\dots,X_r$ espacios separables, entonces su producto es separable.
\end{lem}
\begin{proof}
	Tomamos $A_i$ los conjuntos densos numerables de cada espacio $X_i$. Es claro que el conjunto $\prod_{i=1}^rA_i$ es numerable. Además, por la proposición \ref{const_prop_adherencia} es denso.
\end{proof}
Lindelöf sin embargo no se comporta bien para productos, para ver esto basta un contraejemplo con la topología de Sorgenfrey.
\begin{exa}[Sorgenfrey y productos]
	Sabemos que $\R$ equipado con la topología de Sorgenfrey es Lindelöf, veamos que su producto no lo es. Para ello consideramos un recta oblicua de $\R^2$, que, por ser la topología de Sorgenfrey más fina que la usual, es cerrado. No obstante, la topología relativa asociada a una recta oblicua en el producto de topologías de Sorgenfrey es la discreta, ya que el punto es abierto (¡compruébese!). Por ende, la recta no es Lindelöf, pero Lindelöf se traspasaba a subespacios cerrados, luego $\R^2$ con el producto de topologías Lindelöf no es Lindelöf.
\end{exa}
\subsection{Sumas}
Las sumas, como hasta ahora, conservan todas las propiedades. Veámoslo una a una y sin rodeos.
\begin{lem}[I axioma y sumas]
	La suma de espacios I axioma es I axioma.
\end{lem}
\begin{proof}
	En efecto, dado un punto del espacio suma, este estará en alguno de los estantes. Es decir, será un punto de la forma $(i,x_i)\subset\{i\}\times X_i$. Como $\{i\}\times X_i$ es homeomorfo a $X_i$ vía la inclusión $j_i$ y $X_i$ es primer axioma, podemos tomar una base de entornos (sin pérdida de generalidad abiertos) numerable  de $j_i^{-1}(i,x_i)\in X_i$, digamos $V_k$. Como las inclusiones abiertas, $j_i(V_k)$ es una base numerable de entornos abiertos de $(i,x_i)$.
\end{proof}

\begin{lem}[II axioma y sumas]
	La suma de espacios II axioma es II axioma.
\end{lem}
\begin{proof}
	Sean $\B_i$ bases numerables de los sumandos, demostremos que
	\[\widehat{\B}:=\left\{\{i\}\times U_n^i\midc n\in\N,\ U_n^i\in\B_i\right\}\]
	es base (obviamente numerable) del espacio suma $\sum_{i=1}^rX_i$. En efecto, vemos que todo abierto de la base ``estándar'' es expresable como unión de abiertos de $\widehat{\B}$. 
	\begin{equation*}
		\B=\left\{\{i\}\times\bigcup_{A\subset \N}U_n^i\midc U_n^i\in\B_i\right\}=\left\{\bigcup_{A\subset\N}\{i\}\times U_n^i\midc U_n^i\in\B_i\right\}\qedhere
	\end{equation*}
\end{proof}

\begin{lem}[Separabilidad y sumas]
	La suma de espacios separables es separable.
\end{lem}
\begin{proof}
	En efecto, tomemos los conjuntos $A_i$ numerables y densos en los factores $X_i$. Veamos que el conjunto numerable
	\begin{equation*}
		\bigcup_{i=1}^r\{i\}\times A_i
	\end{equation*}
	es denso en el espacio suma. En efecto, como las inclusiones $j_i$ son inmersiones, si la adherencia de $A_i$ es $X_i$, la adherencia de $\{i\}\times A_i$ es $\{i\}\times X_i$. Como la unión de las adherencias es la adherencia de las uniones el resultado se sigue. 
\end{proof}

\begin{lem}[Lindelöf y sumas]
	La suma de espacios Lindelöf es Lindelöf
\end{lem}
\begin{proof}
	Dado un recubrimiento por abiertos $\Gamma:=\{U_i\midc i\in I\}$ de la suma, fijamos un $j\in\{1,\dots,r\}$ y tomamos la imagen inversa por $j_j$ del recubrimiento, lo cual nos dará un recubrimiento abierto de $X_j$.
	
	Como $X_j$ es Lindelöf, podemos extraer un subrecubrimiento numerable \[\Gamma_j:=\{(j_j)^{-1}(U_i)\midc i\in\sigma\subset I\}\]
	
	Tomamos como parte de nuestro subrecubrimiento a los $U_i$ del recubrimiento original que se corresponden con los abiertos del recubrimiento $\Gamma_j$. Iteramos este proceso con cada $j\in\{1,\dots,r\}$ añadiendo cada vez a nuestro subrecubrimiento los abiertos que correspondan.
	
	Al final nos queda un subrecubrimiento numerable del original. Que es numerable es evidente por ser unión finita de conjuntos numerables y que es recubrimiento es consecuencia de que las inclusiones son inmersiones. Además, es un subrecubrimiento por la propia construcción.
\end{proof}
\subsection{Tabla de comportamiento topológico}
A modo de recapitulación presentamos la siguiente tabla.
\begin{table}[h]
	\centering
	\begin{tabular}{c|c|c|c|c|}
		\cline{2-5}
		\multicolumn{1}{l|}{}           & \multicolumn{1}{l|}{\textbf{Subespacios}}                                      & \multicolumn{1}{l|}{\textbf{Cociente}} & \multicolumn{1}{l|}{\textbf{Producto}} & \multicolumn{1}{l|}{\textbf{Suma}} \\ \hline
		\multicolumn{1}{|c|}{\textbf{I axioma}}       & Sí                                                                    & No                            & Sí                            & Sí                        \\ \hline
		\multicolumn{1}{|c|}{\textbf{II axioma}}      & Sí                                                                    & No                            & Sí                            & Sí                        \\ \hline
		\multicolumn{1}{|c|}{\textbf{Separable}} & \begin{tabular}[c]{@{}c@{}}Sí, en el caso \\ de abiertos\end{tabular} & Sí*                           & Sí                            & Sí                        \\ \hline
		\multicolumn{1}{|c|}{\textbf{Lindelöf}}  & \begin{tabular}[c]{@{}c@{}}Sí, en el caso\\ de cerrados\end{tabular}  & Sí*                         & No                            & Sí                        \\ \hline
	\end{tabular}
	\caption{Tabla resumen de numerabilidad.}
	\label{Tabla_numerabilidad}
\end{table}
En la tabla anterior, el símbolo * indica que, en esos casos, la propiedad de ser separable o Lindelöf se preserva porque la imagen continua de un espacio separable (Lindelöf) es separable (Lindelöf). En tablas posteriores se usará el mismo símbolo para indicar esto mismo.