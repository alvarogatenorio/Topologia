%Cosas Pendientes de Clara Rodríguez Núñez

%Lo he añadido todo al capítulo de compacidad,
%pero me da cosa borrartelo, así que lo dejo por si acaso.
\section{Compacificación de Alexandroff}

Como hemos podido ver a lo largo de este capítulo, los espacios compactos son espacios muy manejables, con buenas propiedades. ¿No sería una gran noticia que todos los espacios con lo que vamos a trabajar lo fueran? Lamentablemente esto no va a ser así, pero si que vamos a lograr en muchas ocasiones hacer de nuestro espacio un subespacio de un espacio compacto.


La idea de las compactificaciones es lograr esto (convertir espacios no compactos en subespacios de compactos) añadiendo puntos al espacio original.
Veamos en primer lugar una definición de compactificación antes de pasar a ver la que da nombre a esta sección.

\begin{defi}[Compactificación]
Llamaremos \tbi{compactificación} de un espacio $X$ a un espacio $Y$ si $X$ es homeomorfo a un subespacio denso de $Y$, y $Y$ es compacto.
\end{defi}


La condición de que el espacio sea denso en su compactificación solo tiene como objetivo que ésta no sea innecesariamente grande. 


Vemos antes de pasar a la de Alexandroff un breve ejemplo para terminar de aclarar la definición.
\begin{exa}
	Si $X\subset K$ con $K$ compacto, $\adher X\subset K$ es también compacta ya que se trata de un cerrado en un compacto y por lo tanto sería una compactificación de $X$.
\end{exa}

Ahora pasamos a ver el caso particular de la compactificación de Alexandroff. En primer lugar la definiremos para luego pasar a ver una serie de observaciones que nos llevaran a demostrar que realmente verifica ser compactificación.
\begin{defi}[Compactificación de Alexandroff]
	Sea $(X,\T)$ espacio topológico localmente compacto, no compacto que verifica ser $T_2$.
	
	Tomamos un punto al que pasamos a denotar como $\infty$ tal que  $\{\infty\}\notin X$ y consideramos $X^*=X\cup\{\infty\}$. 
	
	Tomaremos como  topología la qe viene dada por la siguiente definición:
	\begin{equation}
		\T ^* = \T \cup \{A \subset X^* \colon X^*\setminus A \textup{ es compacto en }X\}
	\end{equation}
	
	Vemos ahora una propiedad que verificará la topología que acabamos de definir,tenemos que  $\infty\in\U\in\T^* \sii X^*\setminus\U$ es compacto en $X$ $[\ra \U\cap X\in\T]$.
\end{defi}
Hemos definido de este modo la \tbi[compactificación!de Alexandroff]{compactificación de Alexandroff} aunque no hemos probado aún que $X*$ sea un espacio compacto ni que su topología este bien definida siquiera, pasamos a hacerlo ahora. Para ello vamos a comprobarlo como una serie observaciones que probaremos inmediatamente.

\begin{obs}En primer lugar enunciaremos en cada apartado una propiedad, probandola a continuación dentro del mismo apartado. Los primeros buscan probar que realmente es topología, luego estudiamos propiedades de la misma.
	 
	\begin{enumerate}
	\item $\emptyset,X^*\in \T^*$, la comprobación en ambos casos es trivial. 
	\item La unión de abiertos de $\T^*$ es abierta. Veamos su demostración.
	
	Podemos descomponer los abiertos de esta unión en dos grupos, los abiertos que contienen a $\infty$ y los que no. De este modo, expresamos la unión como 
	\begin{equation}
		\bigcup_{U\in\T}U_i \cup \bigcup_{\infty\in\W_j}W_j
	\end{equation}
	Denotaremos $\bigcup_{U\in\T}U_i$ como $U$ y $\bigcup_{\infty\in\W_j}W_j$ como $W$. Vayamos ahora por casos.
	
	Si solo tenemos $U$ ($W=\emptyset$) ya lo tenemos, dado que se trataría de unión de abiertos de $\T$ y al ser ésta topología será un abierto de $\T$ y por lo tanto también de $\T^*$.
	
	
	Si solo tenemos $W$ tenemos que $X^*\setminus W=\bigcap_j X^*\setminus W_j$ que será cerrado en $X$ y estará contenido en $X^*\setminus W_{j_0}$ que es compacto por el modo en que hemos definido los $W_{j_0}$.Por lo tanto, tenemos que  $X^*\setminus W=\bigcap_j X^*\setminus W_j$ es un cerrado dentro de un compacto y esto, al ser $X$ Segundo Axioma, implica que es compacto. Con esto, por la definición, tenemos que $W$ es abierto.
	
	Por último, si tenemos tanto $U$ como $W$ no vacíos, la demostración será similar al caso anterior. 
	
	Tomamos $G$ la unión de $W$ y $U$ entonces tenemos $X\setminus G= (X^*\setminus W)\cap(X\setminus U)\subset X^*\setminus W$.
	Como hemos visto, $(X\setminus U)$ es cerrado y $(X^*\setminus W)$ compacto, por lo que la intersección de ambos será un cerrado en un compacto, y por lo tanto compacta. Así, $X\setminus G$ será compacto y por lo tanto $G$ abierto.
	
	\item Veamos que la intersección finita de abiertos es abierta, dados dos abiertos $U$ y $V$ tenemos que, su intersección será abierta. Para probarlo nos basta con observar las tres situaciones distintas que pueden darse:
	\begin{equation}
	U\cap W=
	\left\{ \begin{array}{lcc}
	\infty\notin U,V &  \ra  & \textup{resulta trivial (nos encontramos en }\T)\\
	\\\infty\in U,\infty\notin V &   \ra  & \textup{como $U\cap X\in\T$, asi que ambos son abiertos de $\T$}\\
	\\\infty\in U,V &  \ra  & \textup{ya que la intersección finita de compactos es compacta}\\
	\end{array}
	\right.
	\end{equation}
	Con estos 3 puntos hemos demostrado que $\T^*$ es topología.
	\item Como se desprende de su definición (dejamos al lector el comprobarlo por su cuenta, no hay que recurrir a más que la propia definición) tenemos que $\T^*|_X=\T$. 
	\item Es $T_2$. Dado que como se nos ha dicho $\T$ es $T_2$, tan solo podríamos encontrar problemas a la hora de separar $x\in X$ y $\infty$.
	
	Ahora bien, como $X$ es localmente compacto, $\exists K$ compacto tal que $x\in U\subset K$ siendo $U$ abierto. Además, $X^*\setminus K=W$ es abierto en $\T*$ por la definición de la misma.
	
	Como $U\cap W\subset K\cap (X\setminus K)=\emptyset$ hemos encontrado los entornos de los puntos que buscabamos. 
	
	
	\item $X^*$ es compacto. Gracias a las observaciones desarrolladas anteriormente nos va a ser sencillo probarlo.
	
	Tenemos que dado un recubrimiento por abiertos de $X$ $U_{i\in I}$ tenemos que $\exists U_{i_0}\ni\infty \ra X^*\setminus U_{i_0}=K$ será compacto en $X$. por lo tanto, al ser $U_{i\in I}$ recubrimiento por abiertos de $K$ y éste compacto, tenemos que $K\subset U_{i_1}\cup\cdots\cup U_{i_r}$ (extraemos subrecubrimiento finito del mismo).
	
	Ahora bien, como $K\subset U_{i_1}\cup\cdots\cup U_{i_r}$ y $X^*\setminus K\subset U_{i_0}$ entonces tenemos que $X^*\subset U_{i_0}\cup  U_{i_1}\cup\cdots\cup U_{i_r}$, con lo que habríamos obtenido un subrecubrimiento finito y probado de este modo su compacidad.
	
	\item Completemos ahora la definición de compactificación viendo que $X\subset X^*$ es abierto denso en $X$. 
	
	Ver que es abierto es trivial mediante la definición de $\T^*$ (dado que $X$ es abierto en $\T$). Ahora, para ver que es denso, sea $U$ abierto no vacio y supongamos que $\emptyset\in\U$ (el caso de que no esté no es necesario analizarlo ya que sería trivialmente abierto de $X$ por la definición). Ahora bien, $U=(X^*\setminus K)$ tiene que cortar a $X$ ya que en caso contrario, $U\cap X=\emptyset\ra K=X$ que no es compacto.
	
	
	\item $X^*$ es única, es decir, es homeomorfa a toda otra compactificación por un punto.
	
	Sea ahora $(Y,\T^{'})$  una compactificación por un punto de $X$, siendo $Y= f(X) \cup \{w\}$ siendo $Y$ compacto $T_2$ y $h$ inmersión, abierta, densa.
	
	Definimos $h : Y \longrightarrow X^*$ mediante $h(w) = \infty$ y $h(f(x)) = x\forall x \in X$. Evidentemente $h$ es biyectiva. Nos queda por ver que es continua, dado que al estar en un compacto en $T_2$ esto implica que es cerrada, y por lo tanto homeomorfismo.
	
	Tomamos $U\subset X^*$, vayamos ahora por casos:
	
	Si $U\subset X$, entonces $g(U)=h^{-1}(U)\subset Y$ que será abierto por la definición de la $f$.
	
	Si $\infty\in U\ra X\cap U=X\setminus K$ siendo $K$ compacto. Entonces como $Y\setminus f(K)=f(X\setminus K)\cup \{w\}=h^{-1}(U)$. De este modo, como $h^{-1}(U)=Y\setminus f(K)$ que es abierto dado que $f(K)$ es compacto cerrado.
\end{enumerate}.
\end{obs}
Con estas observaciones, podemos resumir lo demostrado diciendo que la topología que hemos tomado es realmente topología y compactificación de $X$ y además hemos probado que añadir un único punto caracteriza la topología de Alexandroff.
Para terminar esta sección vamos a ver unos pocos ejemplos de compactificaciones.
\begin{exa}
	Veremos ahora ejemplos de compactificaciones. Los primeros tendrán como objetivo entender el mecanismo de la misma. Después, pondremos ejemplos para aclarar que como hemos visto, dado un conjunto sus compactificaciones por un punto son homeomorfas, pero no sucede al contrario (dada una compactificación los conjuntos de los que puede provenir no tienen porque ser homeomorfos). 
	\begin{enumerate}
		\item Tenemos que $\R^*=\mathcal{S}^1$, $\R^{n*}=\mathcal{S}^n\subset \R^{n+1}$.
		\item $[0,1]\ne(0,1)^*$
		\item $(0,1]^*=[0,1] y [0,1)^*=[0,1]$ pero estos dos subespacios no son homeomorfos entre si.
		\item Siendo $X=[0,1/2)\cup(1,2]$ e $Y=[0,1)$ tenemos que sus compactificaciones son la misma $X^*=Y^*=[0,1]$ pero no son homeomorfos.
	\end{enumerate}
\end{exa}
