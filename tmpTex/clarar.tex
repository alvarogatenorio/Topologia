%Cosas Pendientes de Clara Rodríguez Núñez
%%Espacio producto, falta incluir la suma
\section{Producto de espacios topológicos}
Una vez visto el caso del espacio cociente, pasamos ahora a realizar un desarrollo similar para otras construcciones como son los productos y sumas de espacios topológicos. Se va a realizar un procedimiento análogo en el desarrollo de las mismas al de la anterior sección.

Sean $(X_{i },\T_{i})^{1\le i\le r}$ espacios topológicos y sea $Y=\prod_i X_i$. Tomamos entonces 
\begin{equation}
Y=\prod_i X_i \stackrel{p_i^r}{\longrightarrow}(X_i,\T_i)
\end{equation}
Lo que buscamos en la construcción del espacio producto es de dotar a $Y$ de la topología $\T$ más fina haga continuas todas las aplicaciones $p_i^r$ para $1\le i\le r$. A dichas  $p_i^r$ las conoceremos a partir de ahora como \tbi{proyecciones}.

De este modo, como queremos que $(p_i^r)^{-1}$ sea continua  tenemos que para todo abierto $U_i\in \T_i$ deberá verificarse que $(p_i^r)^{-1}(U_i)=X_1\times\cdots\times U_i\times\cdots X_r$ (todos son los totales excepto en el caso de $X_i$) sea abierto. Es decir, que se cumpla $X_1\times\cdots\times U_i \times\cdots X_r\in \T$

Por lo tanto, vemos que por ser topología ha de verificarse que $(p_i^r)^{-1}(U_i)\cap (p_j^r)^{-1}(U_j)= \cdots\times U_i \times\cdots\times U_j\times \cdots$ sea parte de ella (intersección de abiertos).

Si tomamos ahora $ \{ U_1\times\cdots\times U_r : U_i\in \T_i \}$ veamos que es una base de la topología que pretendemos construir. Esto va a resultar evidente, ya que tan solo debemos comprobar para demostrarlo que al intersecar dos elementos cualesquiera de la base nos queda otro elemento de la base. 

Como dados $U_1\times\cdots\times U_r$ y $V-1\times\cdots\times V_r$ elementos de la base su intersección es $(U_1\cap V_1)\times\cdots\times (U_r\cap V_r)$ y como $(U_i\cap V_i)\in \T_i$, afirmamos que forma parte de la base por la definición de la misma.

Así, una vez que tenemos la base que va a caracterizar nuestra topología producto podemos afirmar que esta viene dada por $\T=\prod_i \T_i$.

\begin{defi}[Topología producto]
Sean $(X_{i },\T_{i})^{1\le i\le r}$ espacios topológicos y sea $Y=\prod_i X_i$. Entonces se conoce como \tbi{topología producto} a la topología $\T$ de $Y$ que viene dada por la base $ \{ U_1\times\cdots\times U_r : U_i\in \T_i \}$ (es decir, el producto de las topologías).
\end{defi}
\begin{lem}[Propiedad Universal]
En la siguiente situación,%añadir figura wiki

Tenemos la siguiente \tbi{propiedad universal del producto} que caracteriza la continuidad de $f$ diciendo que esta será continua si y solo si  $f_i=p_i^r\circ f$ son continuas para todo $i$.

\begin{proof}
	Vemos que si $f$ es continua, como $p_i^r$ son continuas, tenemos que $f_i$ también lo será al ser composición de estas.
	
	Por otro lado, tenemos que dado $W=\prod_iU_i$ abierto en $X$, tenemos que $f^{-1}(W)=\bigcap_i f_i^{-1}(U_i)$ será abierto dado que los $f_i^{-1}(U_i)$ son abiertos por ser $f$ continua. Así, como hemos probado que la imagen inversa de abiertos es abierta, tenemos que $f$ es continua.
	\end{proof}
\end{lem}

Veamos ahora una serie de proposiciones relacionadas con la topología producto, las cuales a pesar de resultar sencillas de comprobar nos proporcionan resultados interesantes.

\begin{prop}
	Las proyecciones son abiertas
	
	\begin{proof}
		Sea $W$ abierto en el espacio producto, veamos que $p_i^r(W)$ es abierto.
		
		Nos basta con comprobarlo con los abiertos de la base, es decir, para los elementos de la forma $ U_1\times\cdots\times U_r : U_i\in \T_i$.  Como  $p_i^r(U_1\times\cdots\times U_r)=U_i$ es abierto podemos concluir la demostración.
	\end{proof}
\end{prop}


\begin{prop}
	La aplicación $X:\longrightarrow\{a1\}\times\cdots\times X_i\times\cdots\times \{ar\}\in X_1\times\cdots\times X_r$ tal que $x\longrightarrow (a_1,\dots,a_r)$ es una inmersión.
	
	\begin{proof}
		Veamos primero como transforma la base de abiertos del inicial en la base del final.
		$U_i\longrightarrow\{a1\}\times\cdots\times U_i\times\cdots\times \{ar\}=\{a1\}\times\cdots\times X_i\times\cdots\times \{ar\}\cap(U_1\times\cdots\times U_r)$
		
		Como observamos esto se verifica y además la aplicación al serlo en los abiertos de las bases verifica ser biyectiva, continua y abierta por lo que se tratará de una inmersión.
	\end{proof}
\end{prop}

Ahora pasamos a ver si somos capaces de construir bases más sencillas para nuestra topología producto, es decir, conformadas por menos abiertos. Igualmente, veremos el modo de construcción de bases de entornos en el espacio producto. No vamos a probar que sean bases realmente en ninguno de los dos casos ya que en ambos casos va a resultar muy sencillo aplicando la caracterización de base, dejándose como ejercicio para el lector de cara al repaso de la misma.
\begin{enumerate}
	\item Dado el espacio producto $(\prod_iX_i,\prod_i\T_i)$ tenemos que $\B=\{W_i\times\cdots\times W_r:W_i\in \B_i$ base de $\T_i\}$ es base de entornos del mismo.
	\item Dado el espacio producto $(\prod_iX_i,\prod_i\T_i)$ y $(a_1,\dots,a_r)\in\prod_iX_i$ tenemos que $\Va^{(a_1,\dots,a_r)}$=$\{V^{a_1}\times\cdots\times V^{a_r}:W_i\in \Va^{a_i}$ base de entornos en $\T_i\}$ es base de entornos del mismo.
\end{enumerate}
