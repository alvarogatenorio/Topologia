%Cosas Pendientes de Iván Prada Cazalla

\section{Base de abiertos}
\begin{defi}[Base de abiertos]
	\label{etop_bases_abiertos}
	Una \tbi[base!de abiertos]{base} de $\T$ es una colección $\B$ de abiertos, tal que todo abierto de $\T$ es unión de abiertos de $\B$
\end{defi}
Veamos que esta definición nos permite reformular las bases de entornos vistas en \eqref{etop_bases_entornos}, y que caracterizaremos de la siguiente manera:


\begin{prop}[Reformulación de bases de entornos]
	Sea $x$ un punto de $\X$ y sea $\B^{x} = \{B \in \B : x \in B\}$, $\B^{x}$ es una base de entornos.
\end{prop}
\begin{proof} Veamos que esto ocurre. Para ello todo entorno de $x$ tiene que contener a alguno de $\B^{x}$. Sea $\V^{x}$ un entorno de $x \ra\ \exists\ \U$ abierto tal que $x \in \U \subset \V^{x}$. Ahora, como $\B$ es base de abiertos $\ra\ \U\ =\ \bigcup_{i\in I}B_i\ \ra\ \exists B_{i} \subset \B^{x}$ tal que $x \in B_{i} \subset \V^{x}$, luego cumple lo buscado.
	
	
	Ahora veamos...%Completar, duda en que quiere hacer conn esto 
\end{proof}

Tras haber visto bases de abiertos \ref{etop_bases_abiertos}, y bases de entornos \ref{etop_bases_entornos}, vamos a plantear el II axioma de numerabilidad, que, aunque será tratado con profundidad en un tema posterior, expone muy bien las ideas recién tratadas.
\begin{defi}[II axioma de numerabilidad]
	\label{etop_2_axioma_num}
	Diremos que $\X$ es \tbi[segundo axioma de numerabilidad]{II axioma} (de numerabilidad) si tiene una base numerable.
	\begin{exa}
		ejemmplos...%Por hacer
	\end{exa}
\end{defi}
\begin{obs}
	Ser II axioma implica ser I axioma. %Prueba
\end{obs}
\begin{prop}
	contenidos...
\end{prop}
\begin{exa}[Topología de Sorgenfray en $\R$]
	contenidos...
\end{exa}
