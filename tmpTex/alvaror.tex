%Cosas Pendientes de Álvaro Rodríguez García

%%%%%%%%%%
% Día 28/2
%%%%%%%%%%

% Después del primer cacho de imágenes directas

Ahora que hemos visto que el caso interesante es que $f$ sea sobreyectiva, consideramos el siguiente diagrama:
\[\xymatrix{
(\X,\T) \ar[r]^{f} \ar[d]_{p} &
(\mc{Y}, f\T) \ar@{<->}[ld]^{\bar{f}} \\
X/\mathord{\sim}
}\]
donde la relación de equivalencia verifica $x\sim y\iff f(x)=f(y)$. Entonces, se verifica que la aplicación $\bar{f}$ entre $X/\mathord{\sim}$ y $(\mc{Y},f\T)$ es una biyección, por cómo hemos definido la relación de equivalencia. Además, tenemos que $p$ es simplemente la aplicación que manda $x\in\X$ a su clase de equivalencia. Ahora, vamos a definir la topología cociente para $X/\mathord{\sim}$ y vamos a comprobar que, con ella, $\bar{f}$ es homeomorfismo.

\begin{defi}[Topología cociente]
Definimos la \tbitop[$\T/\mathord{\sim}$]{cociente} en las condiciones anteriores como:
\[\T/\mathord{\sim} = \{V\midc p^{-1}(V)\in\T\}\]
\end{defi}

Entonces, se verifica que $\bar{f}$ es homeomorfismo. Esto se sigue rápidamente de la propiedad universal de la topología imagen, cuyo esquema aparece en la ecuación \ref{cont_propiedad_universal_top_imagen}. En efecto, nótese que la biyectividad de $\bar{f}$ nos permite afirmar que $p=\bar{f}\circ f$. Entonces, por esta propiedad universal, tenemos que $p$ es continua si y solo si lo es $\bar{f}$. Pero como $p$ lo es, entonces necesariamente lo es $\bar{f}$. Para la implicación en el otro sentido, basta darnos cuenta de que hemos definido la topología cociente como una topología imagen, concretamente la topología imagen por $p$. Entonces, se verifica en el cociente la misma propiedad universal y la implicación es totalmente análoga.

Visto esto, está claro que todos los abiertos de $\T/\mathord{\sim}$ son imágenes por $p$ de abiertos de $\T$ (pero no todas las imágenes de abiertos tienen que estar necesariamente en $\T$). Entonces, definimos:

\begin{defi}[Conjunto saturado]
En las condiciones anteriores, decimos que $W\in\T$ es \tbi[conjunto!saturado]{saturado} si $W=p^{-1}(p(W))$. Es equivalente decir que $[x]\cap W\neq\emptyset\implies [x]\subset W$, o que $x\in W,y\sim x\implies y\in W$.
\end{defi}

Gracias a esta definición, podemos reescribir la topología cociente como:
\[\T/\mathord{\sim} = \{p(W)\midc W\in\T\text{, } W\text{ saturado}\}\]

Además, vamos a dar un nombre a las funciones que envían $X$ en un espacio homeomorfo a un cociente:

\begin{defi}[Identificación]
	Una \tbi{identificación} es una aplicación $f:(X,\T)\to (\mc{Y},\T')$ sobreyectiva, cuando $(Y,\T')$ es homeomorfo a $(X/\mathord{\sim},\T/\mathord{\sim})$ para cierta relación de equivalencia $\mathord{\sim}$.
\end{defi}

La aplicación $f$ con la que hemos estado trabajando es, entonces, una identificación. 

\begin{prop}
	Una identificación $f:(X,\T)\to (\mc{Y},\T')$ es continua.
	
	\begin{proof}
		Como la topología cociente es homeomorfa al espacio $(\mc{Y},\T')$, y hay una aplicación continua de $\X$ en el cociente, entonces hay una aplicación continua de $\X$ en $\mc{Y}$.
	\end{proof}
\end{prop}

Para comprobar si una aplicación es una identificación, podemos usar la siguiente proposición:

\begin{prop}[Condiciones suficientes de identificación]
Sea $f:\X\to\mc{Y}$ una aplicación. Si se verifica alguna de las siguientes condiciones, $f$ es una identificación:
\begin{enumerate}
\item $f$ es una aplicación continua abierta.
\item $f$ es una aplicación continua cerrada.
\end{enumerate}
\end{prop}

Nótese que puede haber identificaciones que no verifiquen ninguna de las condiciones anteriores.

Toda esta construcción abstracta debe servirnos para formalizar todo el concepto de cociente de un espacio. Esta es quizá la idea más importante que se va a ver en toda la asignatura, y es excepcionalmente útil para construir una gigantesca variedad de homeomorfismos y encontrar objetos simples homeomorfos a otros mucho más complejos.

\begin{exa}[Circunferencia unidad]
	Definimos la aplicación:
	\[\begin{split}
	\R&\to\esfera^1 \\
	t&\mapsto e^{2\pi it}=(\cos 2\pi t,\sen 2\pi t)
	\end{split}\]
	y consideramos la relación de equivalencia $s\sim t\iff e^{2\pi is}=e^{2\pi it}\iff s-t\in\Z$. Vamos a ver que la circunferencia unidad $\esfera^1$ es homeomorfa a $\R/\mathord{\sim}$.
	
	Nos queda, pues, el siguiente esquema:
	
	% ESQUEMA
	y el homeomorfismo se reduce a demostrar que la aplicación que manda $f$ a $\esfera^1$ es una identificación. 
	
\end{exa}